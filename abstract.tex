HPC user support teams invest a lot of time and effort in installing
scientific software for their users. A well-established practice is
providing environment modules to make it easy for users to set up their
working environment.  Several problems remain, however: user support
teams lack appropriate tools to manage a scientific software stack
easily and consistently, and users still struggle to set up their
working environment correctly.  In this paper, we present a modern
approach to installing (scientific) software that provides a solution
to these common issues. We show how EasyBuild, a software build and
installation framework, can be used to automatically install software
and generate environment modules.  By using a hierarchical module naming
scheme to offer environment modules to users in a more structured way, and
providing Lmod, a modern tool for working with environment modules, we help
typical users avoid common mistakes while giving power users the
flexibility they demand.
