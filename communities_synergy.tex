\subsection{Communities}

Both \easybuild{} and Lmod are active developped, with development being
highly community-driven. Not only are new features implemented by the core
development teams following requests by their respective user communities, the
users themselves also significantly contribute by reporting bugs, sharing patches
for bugs they fixed themselves, and even by implementing new features themselves
and subsequently sharing them so the implementation can be included in later
releases.

Estimating the size of the \easybuild{} and Lmod communities is difficult.
Roughly a dozen different HPC sites actively contribute to the \easybuild{}, with
several others using it. The \easybuild{} mailing list has about 70 subscribers
to date, so a careful informed guess yields an estimate of over one hundred
users. The \easybuild{} IRC channel, one of the other main community hubs, has
about a dozen of daily active users. HPC sites which are currently actively using
\easybuild{} include the different sites in Flanders (Belgium) that together form
the Flemish Supercomputing Centre, University of Luxembourg, The Cyprus Institute,
Gregor Mendel Institute (Austria), University of Basel (Switzerland), University
of Auckland (New Zealand), J\"ulich Supercomputer Centre (Germany), Idaho
National Lab (US), and University of Warwick (UK), as well as commercial
companies including Bayer (Germany).

For Lmod about 50 people are subscribed to the mailing list, but this
is known to represent at best no more than $20\%$ of the Lmod community. With a
couple of hundred HPC sites already providing Lmod to their users, so the actual
numbers of Lmod users will be multiple thousands. Note that for example TACC is
providing Lmod as the only modules tool, already covering 15.000--20.000 users
over 3 different systems. Lmod has been downloading several thousands of times
from the SourceForce code repository, across all versions.
\kenneth{Robert: can you include a list of HPC sites actively using Lmod? Do
include HPC-UGent already in there ;-)}

\subsection{Synergy}

Recently, a synergy has grown between \easybuild{} and Lmod. Both projects have a
community-oriented vision and highly value flexibility, which appeals to users
from both communities. Since \easybuild{} makes installing scientific software
significantly easier, user support teams can quickly generate lots of module
files, which can become overwhelming for users. This strengthens the need for a
modern modules tool that can efficiently deal with large collections of modules,
which led to the \easybuild{} community reaching out to the Lmod community.
Since then, support for using Lmod as a modules tool and support for hierarchical
module naming schemes has been integrated into \easybuild{}, and further
enhancements are planned (see also Section~\ref{sec:future_work}).

Likewise, Lmod has also been enhanced based on feedback by \easybuild{} users.
For example, the support for \texttt{pushenv} (see
Section~\ref{sec:lmod_pushenv}) was included based on discussions on the Lmod
mailing list by members from the \easybuild{} community. Recent versions of Lmod
have also included significant speed improvements which were triggered by issues
that were uncovered via \easybuild{}, e.g., the \texttt{module --terse avail}
command used by \texttt{eb}, which provides easily parseable output, now no longer
reads the module files to obtain properties, since that information is not
required to compose the terse output text.

This synergy results in significantly better tools on both sides, further
pushing the potentially revolutionary nature of these tools.
