\subsection{Communities}
\label{sec:communities}

Both \easybuild{} and Lmod are active developed, with development being
highly community-driven. Not only are new features implemented by the core
development teams following requests by their respective user communities, the
users themselves also significantly contribute by reporting bugs, sharing patches
for bugs they fixed themselves, and even by implementing new features
and subsequently sharing these enhancements so they can be included in later
releases.

Estimating the size of the \easybuild{} and Lmod communities is difficult.
Roughly a dozen different HPC sites actively contribute to the \easybuild{}, with
several others using it. The \easybuild{} mailing list has about 70 subscribers
to date, so a careful informed guess yields an estimate of over one hundred
users. The \easybuild{} IRC channel, one of the other main community hubs, has
about a dozen of daily active users. HPC sites all over the world are
actively using \easybuild{}, including ones in Europe like the
J\"ulich Supercomputing Centre (JSC), the United Stated and
New Zealand, as well as some commercial companies, e.g., Bayer (Germany).

The Lmod mailing list has about 50 people that are subscribed, but this
is known to represent no more than $20\%$ of the Lmod community at best. With a
couple of hundred HPC sites already providing Lmod to their users, the actual
numbers of Lmod users will be multiple thousands. Note that for example TACC is
providing Lmod as the only modules tool, already covering 15.000--20.000 users
over 3 different systems. Lmod has been downloaded several thousands of times
from the SourceForce code repository, across all versions, and is known to be
used by various HPC sites world-wide, including ones in North America (TACC,
Stanford, Harvard, Idaho National Labs) and Europe (HPC-UGent (Belgium), Arctic
University of Norway, University of Warwick (UK)), and by commercial companies
like Total, among may others.

\subsection{Synergy}

Recently, a synergy has grown between \easybuild{} and Lmod. Both projects have a
community-oriented vision and highly value flexibility, which appeals to users
from both communities. Since \easybuild{} makes installing scientific software
significantly easier, user support teams can quickly generate lots of module
files, which can become overwhelming for users. This strengthens the need for a
modern modules tool that can efficiently deal with large collections of modules,
which led to the \easybuild{} community reaching out to the Lmod community.
Since then, support for using Lmod as a modules tool and support for hierarchical
module naming schemes has been integrated into \easybuild{}, and further
enhancements are planned (see also Section~\ref{sec:future_work}).

Likewise, Lmod has also been enhanced based on feedback by \easybuild{} users.
For example, the support for \texttt{pushenv} (see
Section~\ref{sec:lmod_pushenv}) was included based on discussions on the Lmod
mailing list by members of the \easybuild{} community. Recent versions of Lmod
have also included significant speed improvements which were triggered by issues
that were uncovered via \easybuild{}, e.g., the \texttt{module --terse avail}
command which is used by \texttt{eb} and provides easily parseable output, now no
longer reads the module files to obtain properties, since that information is not
required to compose the terse output text, resulting in significant speedups.

This synergy results in significantly better tools on both sides, further
pushing the potentially revolutionary nature of these tools.
