\markus{Here we should probably stress again that all users benefit from
contributions (especially for EB) $\rightarrow$ collaboration; But since this
is a synergy through community, I'm unsure where to add this...}
\kenneth{handled in 1st paragraph}

\subsection{Communities}
\label{sec:communities}

Active development and vibrant communities characterize both \easybuild{} and Lmod.
Requests and suggestions from their respective community lead to new features
implemented by the core development teams. The users themselves also significantly
contribute by reporting bugs, sharing patches, and even implementing new features that
make their way into the public baselines. This effectively makes both tools
platforms for collaboration, letting all users benefit from enhancements.

Estimating the size of these communities is difficult. Roughly a dozen different HPC
sites actively contribute to \easybuild{}, with several others using it.
The \easybuild{} mailing list has about 70 subscribers to date, so an
estimate of over one hundred users is plausible. The \easybuild{} IRC channel, one of
the other main community hubs, has about a dozen of daily active users. HPC sites all
over the world are using \easybuild{}, including various sites in Europe
(e.g., J\"ulich Supercomputing Centre in Germany)
\markus{The double-parens look ugly. Any ideas how to fix this?} \kenneth{fixed},
Idaho National Laboratory (US),
University of Auckland (New Zealand), etc., as well as some commercial companies
(e.g., Bayer).

The Lmod mailing list has about 50 subscribers, which is no more than $20\%$ of the
Lmod community. Since a couple of hundred HPC sites already deploy Lmod, the actual
number of Lmod users is certainly many thousands. TACC, with 15\,000--20\,000
users on three major systems, provides only Lmod as a modules tool. Lmod downloads
from the SourceForge code repository number several thousand, across all versions.
Lmod is deployed at numerous HPC sites world-wide, including North America (TACC,
Stanford, Harvard, \ldots) and Europe (University of Warwick (UK),
Arctic University of Norway, \ldots)\markus{see above}\kenneth{fixed}, and also has commercial users, including Total.

\subsection{Synergy}

Recently, the \easybuild{} and Lmod teams have developed
a strong and mutually beneficial synergy. Both projects have a
community-oriented vision and highly value flexibility, which appeals to users
from both communities. Since \easybuild{} makes installing scientific software
significantly easier, user support teams can quickly generate many environment
modules, which could become overwhelming for users. This strengthens the need for a
modern modules tool that can efficiently deal with large collections of available
modules, which led to the \easybuild{} community reaching out to the Lmod community.
Since then, support for using Lmod as a modules tool and  hierarchical
module naming schemes has been integrated into \easybuild{}, and further
enhancements are planned (see also Section~\ref{sec:future_work}).

Similarly, Lmod has benefited greatly from feedback by \easybuild{} users.
For example, support for \texttt{pushenv} (see
Section~\ref{sec:lmod_pushenv}) grew out of discussions on the Lmod
mailing list by \easybuild{} stakeholders. Recent versions of Lmod
have also included significant speed improvements triggered by issues
uncovered through \easybuild{}. A particular example is the fact that
the \texttt{module --terse avail} command, which provides machine-readable output and
is used by \easybuild{}, no longer parses module files to obtain module properties,
since this is not necessary to compose the terse output text.

The synergy between \easybuild{} and Lmod results in improved tools on both sides,
and is driving the state of the practice in directions that we hope will
have major impact.
\markus{tone down: significantly better $\rightarrow$ improved; prove truly
revolutionary $\rightarrow$ ??? (can't think of anything better right now, but
`revolutionary' sounds a bit exaggerated to me...)}\kenneth{handled, better now?}
