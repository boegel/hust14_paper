Despite the well-established environment modules system, users of HPC systems keep
running into problems with setting their working environment, due to an overwhelming
amount of modules being available, some of which are incompatible with each other, and
module tools that are inadequate for dealing with these problems. Additionally, HPC
user support teams struggle to maintain a consistent set of module files, next to
dealing with the ubiquitous problem of simply getting scientific software installed.
Although these issues are widely recognized, adequate yet flexible policies and tools
which can help resolve them have been lacking.

In this paper, we presented a modern approach to installing scientific software that
deals with these abundant problems. We explained how using a hierarchical module naming
scheme can avoid users shooting themselves in the foot, but also highlighted the need
for more advanced tools in order to efficiently implement such a policy in a
user-friendly way. Two actively developed and community-driven open-source projects
that provide the necessary features to fullfill this need, \easybuild{} and Lmod, were
discussed in detail. \easybuild{} not only automates the tedious and time-consuming
process of installing scientific software and the accompanying module files, it also
provides full control over important aspects such as the module naming scheme being
used and acts as a platform for collaboration between HPC sites worldwide. Lmod on the
other hand allows end-users to easily navigate a hierarchically organized module
stack, next to offering various other useful capabilities that are missing in the
commonly used Tcl-based module tools.

With the proposed methodology, HPC user support teams can efficiently implement a
hierarchical module naming scheme according to their own site policies, and offer
their users the simple and robust interface they deserve for defining their working
environment.
