Despite the well-established environment modules system, users of HPC systems still
run into problems managing their working environments. The reasons are numerous; key
issues include an overwhelming number of modules on modern systems, avoiding
incompatibilities between some modules, and module tools that cannot meet users' needs 
and expectations. Additionally, HPC user support teams struggle to maintain a
consistent set of module files, not to mention the ubiquitous problem of
installing (scientific) software correctly and repeatably. Although these issues are
widely recognized, suitable practices and tools have been lacking.

In this paper, we have presented a modern approach to installing scientific software
that deals with these problems. We have explained the advantages of
using a module hierarchy, and highlighted the need for more advanced tools to
efficiently support this in a user-friendly way. We discussed in detail two
actively developed and community-driven open-source projects that provide the
necessary features to address this need, \easybuild{} and Lmod. \easybuild{} not only
automates the tedious and time-consuming process of installing scientific software
and the accompanying module files, it also provides full control over important
aspects such as the module naming scheme being used. \easybuild{} acts as a platform
for collaboration between HPC sites worldwide. Lmod on the other hand allows end-users
to easily navigate a hierarchically organized module stack, and delivers a variety of
other useful features missing in the commonly used Tcl-based module tools.

Together, \easybuild{} and Lmod allow HPC user support teams to efficiently implement
and tailor a hierarchical module naming scheme that truly meets their requirements,
and offers end users the simple and robust interface they deserve for managing their
working environments.
