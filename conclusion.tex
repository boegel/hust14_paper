Despite the well-established environment modules systems users of HPC systems keep
struggling to set up their working environment, due to an overwhelming amount of
modules being available and tools to interact with these modules that are no longer
evolving. Incompatibilities between modules, be it explicit via conflicts included
in module files or not, result too often in users getting stuck. This puts pressure on
HPC user support teams, who already have issues with maintaining a consistent set of
modules and dealing with the ubiquitous problem of simply getting scientific software
installed. Although these issues are widely recognized, adequate yet flexible policies
and tools which can help resolve this issue are lacking.

In this paper, we presented a modern approach to installing scientific software that
deals with these abundant problems. We explained how using a hierarchical module naming
scheme can avoid users shooting themselves in the foot, but also highlighted the need
for more advanced tools in order to efficiently implement such a policy in a
user-friendly way. We presented two actively developed and community-driven open-source
projects, \easybuild{} and Lmod, that provide the necessary features to do deal with
this. Not only can \easybuild{} automate the tedious and time-consuming process of
installing scientific software and the accompanying module files, it also provides
full control over important aspects such as the module naming scheme being used and
acts as a platform for collaboration between HPC sites worldwide. Lmod on the other
hand allows end-users to easily navigate a hierarchically organized module stack, next
to offering various other useful capabilities that are missing in the traditional
module tools which are no longer actively developed.

With these modern tools in hand, HPC user support teams can effectively implement a
hierarchical module naming scheme according to their own site policies, and offer
their users the simple and robust setup they deserve for defining their working
environment.
