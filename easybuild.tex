\easybuild{}~\cite{EasyBuildSC12} is a software build and installation framework
written in Python. Its primary goal is to alleviate the ubiquitous burden of
building and installing scientific software~\cite{Dubois03}.

The \easybuild{} project was started in 2009 by the HPC team of Ghent
University (Belgium), out of frustration over the lack of appropriate tools for
dealing with the installation of scientific software. The first public release of
\easybuild{} (version 0.5) occurred in April 2012 under an open source license
(GPLv2), after 2.5 years of in-house
development. In November 2012, \easybuild{} v1.0 was released featuring
a stable API. Under a `release early, release often' strategy with a new major
release every 4 to 6 weeks, the development team has a strong focus on continuous,
high quality maintenance, support, and improvement. At the time of
writing, the latest release is \easybuild{} v1.14.0 (July 2014).
\markus{Update?}
\markus{Add ``Currently, \easybuild{} only supports x86/Power-based Linux clusters,
however, support for Cray X-series systems is under development.''?}

Under the motto ``\emph{building software with ease}", \easybuild{} provides
\markus{tone down $\rightarrow$ aims to provide}
an easy
yet powerful way to automatically install (scientific) software stacks, in a robust,
consistent and reproducible way. Its design is deliberately made very flexible and
modular (see Sections~\ref{sec:eb_configurability} and \ref{sec:eb_extensible})
making it an ideally suited \markus{tone down $\rightarrow$ a well suited}
platform for collaboration across HPC sites, as is
confirmed by the steadily growing \easybuild{} community (see
Section~\ref{sec:communities}).

\subsection{Concepts and design}

\easybuild{} consists of a collection of Python modules and packages that
interact with each other, dynamically picking up additional Python modules as needed
for building and installing a (stack of) software package(s) specified via simple
specification files. Or, in \easybuild{} terminology: the \easybuild{}
\emph{framework} leverages \emph{easyblocks} to automatically build and install
software using a particular \emph{compiler toolchain}, as specified by one or
multiple \emph{easyconfig files}.

\subsubsection{\easybuild{} framework}
\label{sec:eb_framework}

The \easybuild{} framework embodies the core of the tool,
providing functionality commonly needed when installing scientific software on HPC
systems. For example, it deals with downloading, unpacking, and patching of sources,
loading module files for dependencies, setting up the build enviroment, autonomously
running (interactive) shell commands, creating module files that match the
specification files, etc.

Included in the framework is an `abstract' implementation of a software build and
install procedure, which is split up into different \emph{steps}: unpacking sources,
configuration, build, installation, module generation, etc. Most of these steps,
i.e., the ones that are generally more-or-less analogous across
different software packages, have appropriate (default) implementations. The only
exceptions are the configuration, build and installation steps that are purposely
left unimplemented (since there is no common procedure for them). Each of the steps
can be tweaked and steered via different parameters known to the framework, for
which values are either obtained from the provided specification files (see
Section~\ref{sec:eb_easyconfigs}) or set to reasonable default values.

In \easybuild{} v1.14.0 the framework source code consists of about 18\,000 lines
of code, organized across about 125 Python modules in roughly a dozen Python
package directories, next to over 6\,000 lines of code for tests. This provides some
notion of the size of the \easybuild{} framework and the amount of supporting
functionality it has to offer.
\markus{Update?}

\subsubsection{Easyblocks}
\label{sec:eb_easyblocks}

The implementation of a particular software build and install procedure is done in
a Python module, which is aptly referred to as an `easyblock'. Each easyblock
ties in with the framework API by defining (or extending/replacing) one or more of
the step functions that are part of the abstract procedure used by the
\easybuild{} framework. Easyblocks typically heavily rely on the supporting
functionality provided by the framework, for example for (autonomously)
executing (interactive) shell commands and obtaining the command output and
exit code.

A distinction is made between \emph{software-specific} and \emph{generic} easyblocks.
Software-specific easyblocks implement a build and install procedure which is
entirely custom to one particular software package (e.g., WRF), while
generic easyblocks implement a procedure using standard tools (e.g., CMake or
make). Since easyblocks are implemented in an object-oriented scheme,
the step methods implemented by a particular easyblock can be reused in others via
inheritance, enabling code reuse across build procedure implementations
\markus{, a major advantage over script-based solutions}.

For each software package being built, the \easybuild{} framework will determine
which easyblock should be used, based on the name of the software package or the
value of the \texttt{\small easyblock} specification parameter. In case an easyblock
specification is not provided and no (software-specific) easyblock matching the
software name could be found, a fallback mechanism will resort to using the generic
\texttt{\small ConfigureMake} easyblock, which implements the common
\texttt{\small configure} -- \texttt{\small make} -- \texttt{\small make install}
procedure.

At the time of writing, the most recent release of \easybuild{} included 137
software-specific easyblocks and 20 generic easyblocks \markus{Update?}, providing support for
automatically installing a wide range of software packages. Examples range
from fairly easy-to-build programs like gzip, over commonly used \markus{commonly used $\rightarrow$ basic/essential?} tools like
compilers, various MPI stacks and commonly used libraries, to large scientific
software packages that are notorious for their involved and painful install
procedures, such as CP2K, NWChem, OpenFOAM, QuantumESPRESSO, and WRF.
\markus{One reviewer suggested to add a reference for OpenFOAM
when it's first use (I removed this usage above ;-)). Do you see a need to add
more references here? I don't...}

\subsubsection{Compiler toolchains}
\label{sec:eb_toolchains}

\easybuild{} also employs so-called `compiler toolchains' (or simply `toolchains'
for short), which were already mentioned in Section~\ref{sec:Module_naming_scheme}.
A toolchain consists of a (set of) compiler(s), usually together with some
libraries for specific functionality, e.g., for using an MPI stack for distributed
computing, or which provide optimized routines for commonly used math operations,
e.g., the well-known BLAS/LAPACK APIs for linear algebra routines. For each software
package being built, the toolchain to be used must be specified
(see Section~\ref{sec:eb_easyconfigs}).

The \easybuild{} framework prepares the build environment for the different
toolchain components, by loading their respective modules and defining environment
variables to specify compiler commands (e.g., via \texttt{\small \$F90}), compiler
and linker options (e.g., via \texttt{\small \$CFLAGS} and
\texttt{\small \$LDFLAGS}), the list of library names to supply to the linker (via
\texttt{\small \$LIBS}), etc. This enables
making easyblocks largely toolchain-agnostic since they can simply rely on these
environment variables; that is, unless they need to be aware of, for example, the
particular compiler being used to determine the build configuration options.

Recent releases of \easybuild{} include out-of-the-box toolchain support for: various
compilers, including GCC, Intel, Clang, and CUDA; common MPI libraries such as
Intel~MPI, MPICH2, MVAPICH2, and Open\,MPI; various numerical libraries including
ATLAS, Intel MKL, OpenBLAS, and ScaLAPACK; and libraries providing FFT routines
like FFTW.

\subsubsection{Easyconfig files}
\label{sec:eb_easyconfigs}

The specification files that are supplied to \easybuild{} are referred to as
`easyconfig files' (or simply `easyconfigs'), which are basically plain text
files containing (mostly) only key-value assignments for build parameters supported
by the framework, also referred to as `easyconfig parameters'. Some
parameters are mandatory, like \texttt{\small name} and \texttt{\small version} to
specify which software (version) should be installed, \texttt{\small
toolchain} to indicate which compiler toolchain should be used, etc. Others are
optional, and have appropriate defaults set for them in the \easybuild{} framework;
examples include \texttt{\small buildopts} to specify options for the build command, 
and \texttt{\small dependencies} to list the software dependencies that should be
taken into account. Note that easyconfig files only provide the bits of information required
to determine the corresponding module name; the module name itself is computed
by \easybuild{} framework by quering the module naming scheme being used, see also
Section~\ref{sec:eb_hmns}. The complete list of supported easyconfig parameters can
be easily obtained via the \easybuild{} command line.

As such, each easyconfig file provides a complete specification of which particular
software package should be installed, and which settings should be used for building
it. After completing an installation, \easybuild{} copies the used easyconfig file
to the install directory, and also supports maintaining an easyconfig archive which
is updated on every successful installation. Therefore, reproducing installations
becomes trivial.

The latest release of \easybuild{} includes over 2\,500 easyconfig files, for
500 different software packages. \markus{Update?}

\subsection{Basic usage}
\label{sec:eb_basic_usage}

As the name suggests, \easybuild{} is \markus{intended to be} particularly easy to use. A script aptly named
\texttt{\small eb} is provided to interact with the \easybuild{} framework.
Launching \texttt{\small eb} with an easyconfig file as an argument triggers a
series of events. First, the easyconfig file will be parsed by the \easybuild{}
framework to determine which software package needs to be installed, and which
easyblock and build parameters should be used. Afterwards, the environment is
prepared by loading the modules for the specified toolchain and dependencies---if
those are available---and the toolchain support in the framework defines additional
environment variables as discussed in Section~\ref{sec:eb_toolchains}. In case the
required modules are not available, an appropriate error message is shown, unless
the automatic dependency resolution mechanism is enabled via the command line option
\texttt{\small --robot} (see Section~\ref{sec:eb_dependency_resolution} for more
details). Next,
the appropriate build and install procedure is executed step by step, as defined by
the framework and the selected easyblock. Finally, if the installation was successful, a module file is generated that
matches the used specifications in terms of module name and contents.

For example, the following command line instructs \easybuild{} to install WRF v3.5
and its missing dependencies with the \texttt{\small goolf} toolchain v1.6.10, as
specified by the provided easyconfig file:

{\small
\begin{alltt}
    \textbf{\% eb WRF-3.5-goolf-1.6.10.eb --robot}
\end{alltt}
}
\noindent
Further details on using \texttt{\small eb} are beyond the scope of this paper; we
refer to \texttt{\small eb --help} and the \easybuild{} wiki~\cite{ebwiki} for
extensive documentation.

\subsection{Feature highlights}
\label{sec:eb_features}

We now briefly highlight the key features of \easybuild{} that are relevant
to the topic of this paper, including automatically resolving dependencies and
the flexibility of the \easybuild{} framework. Other interesting features which are
not discussed here include, but are not limited to, thorough logging of the build
and install process, support for tweaking provided easyconfig files directly from
the \texttt{\small eb} command line, and off-loading individual installations to
a cluster resource manager like PBS.

\subsubsection{Automatic dependency resolution}
\label{sec:eb_dependency_resolution}

\easybuild{} supports installing a entire software stack, including the required
toolchain if needed, with a single \texttt{\small eb} invocation. By enabling the
\texttt{\small --robot} command line option, the dependency resolution mechanism will
construct a full dependency graph for the software package(s) being installed, after
which a list of dependencies is composed for which no module is available yet. Each of
the retained dependencies will then be built and installed, in the required order as
indicated by the dependency graph. This is particularly useful for software
packages that have an extensive list of dependencies, or when reinstalling software
using a different compiler toolchain.

\subsubsection{Configurability}
\label{sec:eb_configurability}

Allowing users to modify \easybuild{}'s default behavior to their needs is another
important feature. \easybuild{} can be configured in three different ways: via one or
more configuration files, via environment variables prefixed with
\texttt{\small EASYBUILD\_}, and via command line options. For each command line
option, a matching environment variable can be set or matching setting can be
defined in a configuration file. Command-line options overrule environment
variables, while they in turn take precedence over configuration files, resulting in a
flexible way of controlling \easybuild{}'s behavior.
Configuration parameters are available for straightforward aspects such as the
installation prefix for software and module files, but also, for example, for the
log level, the modules tool, and the module naming scheme that should be used.

\subsubsection{Dynamic extensibility}
\label{sec:eb_extensible}

Another key feature is that \easybuild{} can be extended dynamically, i.e., the
framework is designed such that additional functionality can be easily plugged in:
additional easyblocks, support for more compilers and libraries to be part
of a toolchain, custom module naming schemes, etc. Extending \easybuild{} is done by
implementing a Python module in the required namespace (e.g.,
\texttt{\small easybuild.easyblocks}), and modifying the \texttt{\small \$PYTHONPATH}
environment variable to make sure that it is available in the Python search path at
runtime. Every time \texttt{\small eb} is used, the framework will `scan' the Python 
search path to determine the available options for each of the dynamically 
extendible aspects. This provides \easybuild{} users the freedom to experiment with
functionality which is not (yet) available in the particular version they are using,
or to extend the supported options for different aspects of \easybuild{} with
additional site-specific options, e.g., easyblocks for in-house software packages or 
a custom module naming scheme.

\subsection{Support for hierarchical module naming schemes}
\label{sec:eb_hmns}

Since \easybuild{}~v1.14.0, sufficiently fine-grained control of the active module
naming scheme is available to support custom hierarchical
module naming schemes. As discussed in Section~\ref{sec:eb_features}, it suffices to
implement the specific details of the hierarchical module naming scheme in a Python
module, which must live in the \texttt{\small easybuild.tools.module\_naming\_scheme}
namespace. In particular, the Python module must provide a Python class that derives
from the `abstract' \texttt{\small ModuleNamingScheme} class provided by the
\easybuild{} framework, and defines a number of methods that cover different aspects
of the module naming scheme. For example, the
\texttt{\small det\_modpath\_extensions} method must return a list of strings
representing \texttt{\small \$MODULEPATH} extensions, given a
parsed easyconfig file as an argument. This is a simple yet powerful approach,
providing full control over all particular aspects of the module naming scheme.
With this in place, \easybuild{} must be configured to use this particular module
naming scheme, see also Section~\ref{sec:eb_configurability}.

Letting \easybuild{} automatically generate module files under a specified
(hierarchical) module naming scheme stands in stark contrast with the practice of
manually creating module files while trying to consistently apply site policies. It avoids the
various issues involved with manually creating module files with respect to consistency and
correctness which were disscussed in Section~\ref{sec:traditional}, and as such
mostly relieves user support teams from another tedious task related to installing
scientific software.
