
\remark{this section should outline the modern approach of using a hierarchical module naming scheme,
highlighting how this makes things easier for end-users without limiting power users; forward references should
be made to the EasyBuild \& Lmod sections when the consequences of using a module naming scheme like this are discussed}

The idea behind a hierarchical module organization is to make different
variants of module files only visible on demand. That is, initially only a
small number of core modules are available to the user. Core modules are
modules which are independent of a particular compiler toolchain, that is,
either completely self-contained or only dependent on basic system software.
Examples of this category are compiler modules, modules providing statically
linked software (e.g., debuggers), and modules providing software packages
using only (certain) scripting languages.

Compiler modules implicitly extend the \texttt{MODULEPATH} to make all
modules visible which are built with---and therefore depend on---the
corresponding compiler. That is, separate software installation and module
files directories are maintained for each version of each compiler. For the
Open\,MPI example presented in Section~\ref{sec:Module_files}, this means
that the user then only sees a single module file for Open\,MPI 1.7.3 instead
of three, the one enabling the installation built using the compiler module
that is currently loaded.

Obviously, such a hierarchy is not limited to a single level. For example,
the module files for different MPI implementations can further extend the
\texttt{MODULEPATH} to enable all modules which depend on both the loaded
compiler and the respective MPI installation.


\label{sec:hierarchical_advantages}
\subsection{Advantages over traditional module naming schemes}

\markus{This sub-subsection IMHO only makes sense if there would be a second
one. I propose to remove it and simply list the advantages as part of Section
III.A.}

A hierarchical module file organization has a number of significant
advantages. First, at any point in time, users are only presented with the
modules which are meaningful in the current context. That is, the list of
available modules is much shorter and therefore less overwhelming. Second,
encoding the dependency chain in the module name is no longer necessary for
modules enabling another hierarchy level, thereby leading to more intuitive
module names. And finally, loading of incompatible modules is automatically
avoided, preventing users from making simple mistakes which may lead to
subtle errors that are non-obvious and hard to debug.

\markus{Somewhere, we should mention that the hierarchical module scheme as
outlined above \textit{in principle} also works with C/Tcl or Tcl-only
modules, but that specific features simplify usage a lot: ``stack-based'' swap,
module deactivation, \texttt{module spider}, etc. But there is a small
chicken/egg problem here \ldots}

\remark{basic users make less mistakes, expert users retain full freedom}

\label{sec:hierarchical_consequences}
\subsection{Consequences of using a hierarchical naming scheme}
