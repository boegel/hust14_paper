%\remark{this section should outline the modern approach of using a hierarchical module naming scheme,
%highlighting how this makes things easier for end-users without limiting power users; forward references should
%be made to the EasyBuild \& Lmod sections when the consequences of using a module naming scheme like this are discussed}

The use of \emph{hierarchical naming schemes} is an excellent way to
help users avoid the pitfalls described above. This approach
makes it possible to organize environment modules in a more structured
way. The key idea is to make modules available in a step-by-step fashion as
their dependencies become part of the user environment.  Initially only a small
number of so-called \emph{core} modules are available to the user. Core modules are
those that do not depend on software choices available to the user; they are either
completely self-contained or depend only on basic system software. Examples include
modules for compilers, and statically linked software like debuggers.

These core modules extend the module search path
(\texttt{\small \$MODULEPATH}) when loaded to make additional modules visible, for
example the ones which are built with---and therefore depend on---the currently loaded
compiler. Separate sub-directories for each version of each 
compiler store the software and module files that depend on
that compiler. For the
Open\,MPI example presented in Section~\ref{sec:Module_naming_scheme}, this means
that the user sees only the module for OpenMPI 1.7.3 that depends
the current compiler of choice.

{\small
\begin{alltt}
    \textbf{% module avail}
    ------------ \emph{<prefix>}/Core ------------
    GCC/4.8.2   Intel/14.0  Clang/3.4
    \textbf{% module load GCC/4.8.2}
    \textbf{% module avail}
    ------------ \emph{<prefix>}/Core ------------
    GCC/4.8.2   Intel/14.0  Clang/3.4
    ----- <\emph{prefix}>/Compiler/GCC/4.8.2 -----
    OpenMPI/1.7.3\
\end{alltt}
}
\noindent
Such a hierarchy is not limited to a single level. The
module files for each MPI implementation, for example, can further extend
\texttt{\small \$MODULEPATH} to make visible the modules that depend on the currently loaded compiler and MPI stack.


\subsection{Advantages over traditional module naming schemes}
\label{sec:hierarchical_advantages}

%\markus{This sub-subsection IMHO only makes sense if there would be a second
%one. I propose to remove it and simply list the advantages as part of Section
%III.A.}

Using a hierarchical module naming scheme has a number of important
advantages. First, at any point in time, users see only the
modules which are meaningful in the current context. That is, the list of
available modules is much shorter and grouped by level of the hierarchy. This is easier for users to process when hundreds of modules are
provided (which is generally the case on large HPC systems).

Second, encoding the dependency chain in the module name is no longer
necessary. This leads to significantly shorter and more intuitive
module names, usually consisting of the software name and version, e.g.,
\texttt{\small WRF/3.5} instead of the long and cryptic module names shown in
Section~\ref{sec:Module_naming_scheme}.

Third, loading incompatible
modules is no longer possible, which eliminates a broad class of subtle errors that
are difficult to debug. This not only dramatically improves the user experience; it
also significantly reduces the time a user support team needs to spend on related
problems.

Finally, this structured organization of module files provides
a number of new opportunities to enhance the user experience. One significant
example is module \emph{swapping}: when the user chooses a new compiler or MPI
stack by executing a command like \texttt{\small module swap gcc intel}, the module
system could (and should!) automatically replace lower-level modules with new
versions compatible with the user's new higher level selections. We strongly believe
that this capability should be part of any robust module system; see
Section~\ref{sec:using_a_hierarchy}.

%\remark{basic users make less mistakes, expert users retain full freedom}

\subsection{Using a hierarchical module naming scheme}
\label{sec:using_a_hierarchy}

In theory, the commonly used Tcl/C and Tcl-only environment modules tools
both support the concept of hierarchical modules; this is because these tools function in a way that is essentially independent of any particular choice in organizing the module files. In practice, this presents some difficulties when using these tools with a software hierarchy.

\subsubsection{Visibility of modules}
\label{sec:hierarchical_consequences_visibility}

In the hierarchical scheme we describe above, a module is not available (visible)
to the user until that module's dependencies are part of the user environment.  This
is by design, and has great advantages. But a module system should not leave to the
user the burden of locating needed modules within a complicated directory structure.  Instead, the module system should provide a built-in, natural mechanism for exposing hierarchies and
dependencies, and make it possible for users to load needed modules
without resorting to raw searches of the system's directory structure. In particular, the module system must provide a means for
displaying relevant information about modules that are outside the 
currently loaded \texttt{\small \$MODULEPATH}.

\subsubsection{Awareness of module path extensions}
\label{sec:hierarchical_consequences_extensions}

The modules tool should be aware of the changes  to the module path that
occur when loading modules in a hierarchical scheme. This is what makes
automatic module swapping possible. When the user executes \texttt{\small module swap} to replace one module with another, the tool needs to i) detect that a particular
module depends on the module being swapped out (this requires an awareness of
the hierarchical structure); ii) unload the dependent
modules and iii) then afterwards
automatically (try to) replace them with equivalent compatible modules, taking into
account the correct order in which to reload modules.

\subsubsection{Module availability on different paths in the hierarchy}
\label{sec:hierarchical_consequences_availability}

The modules tool also needs to take into account the fact it may not always be
possible to reload all dependent modules after swapping modules. When swapping
one compiler for another for example, it is possible that no compatible version of a
lower-level module is available. Unless the module system can manage this issue
seamlessly, the user could end up stuck with a broken environment somewhere between
the original state (before the swap) and the intended end state (in which all
dependent modules are reloaded). Ideally, the module tool should notify the user
when dependent modules cannot be reloaded. Moreover, if the modules tool can keep
track of which modules failed to reload it can also support the ability to revert
the swap, and restore the original set of loaded modules.

The existing Tcl/C and Tcl-only modules tools currently do not provide any of this
functionality. At one point, the Tcl-only tool did briefly support reloading
dependent modules after a \texttt{\small module swap}, but this capability later
disappeared because of incompatibility with the Tcl/C tool.

% moved to lmod.tex

%Finally, \remark{saved collections to set default compiler/MPI to avoid that
%novice users need to pick a compiler/MPI stack before they can get anything
%done}
%\kenneth{Robert: does this make sense to include it? or is this not
%related enough to a hierarchical scheme?}
%\robert{I would remove from Finally ... to here.  It is a mixing of
%  concepts. At TACC we provide a default set of modules which the user
%  can override with their own personal collection of modules}

%\markus{Somewhere, we should mention that the hierarchical module scheme as
%outlined above \textit{in principle} also works with C/Tcl or Tcl-only
%modules, but that specific features simplify usage a lot: ``stack-based'' swap,
%module deactivation, \texttt{module spider}, etc. But there is a small
%chicken/egg problem here \ldots}

\subsection{Maintaining a module hierarchy}
\label{sec:maintaining_a_hierarchy}

Next to the important issues related to using a module hierarchy discussed in the
previous section, an additional consideration is the potential impact on the user
support staff maintaining the modules.

In particular, a great deal of care must be exercised in constructing module files
for a hierarchical software stack; failing to do so can produce unexpected results
and inconsistent environments. One must think carefully about the necessary module
path extensions required at different levels of the hierarchy, take into account
dependencies, and the order in which users should load modules. Additionally, the
module files must be designed such that they use the short name visible to users
(e.g., \texttt{\small WRF/3.5}) rather than the `full' name relative to the top of
the hierarchy (e.g., \texttt{\small MPI/GCC/4.8.2/OpenMPI/1.7.3/WRF/3.5}).

Creating module files manually is already a tedious task; doing so by hand in a
hierarchical context only further complicates this. It should be clear that a modern 
HPC environment requires powerful, flexible, and reliable tools and techniques that
truly \emph{automate} the task of installing scientific software.

%% \subsection{Tools}
%%   \label{sec:tools_for_hierarchical}
In the sections that follow we describe how two community-driven tools work
together to achieve this task: \easybuild{}, a software build and installation
framework, and Lmod, a modern alternative to the Tcl-based environment modules
tools. Because our experience confirms the substantial benefits of a hierarchical
module naming scheme, and because we strongly believe that the HPC community is
ready to (and also should) move in this direction, we focus primarily on the aspects
of these tools that support this approach.
