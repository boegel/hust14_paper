
\remark{this section should outline the modern approach of using a hierarchical module naming scheme,
highlighting how this makes things easier for end-users without limiting power users; forward references should
be made to the EasyBuild \& Lmod sections when the consequences of using a module naming scheme like this are discussed}

A solution to the common pitfalls that users run into when setting up their
working environment using modules (see Section~\ref{sec:issues_traditional})
is using a so-called \emph{hierarchical module naming scheme}, in which the
module files are organized in a more structured way.  The key idea behind 
this is to make module files only available in a step-by-step fashion.  That
is, initially only a small number of so-called \emph{core} modules are
available to the user.

Core modules are modules which are independent of a particular compiler
toolchain, that is, they are either completely self-contained or only
dependent on basic system software. Examples include module files for
compilers and statically linked software (e.g., debuggers).

These core modules (implicitly) extend the module search path
(\texttt{\$MODULEPATH}) to make additional modules visible, for example
the ones which are built with---and therefore depend on---the corresponding
compiler. As such, separate software installation and module files
subdirectories are maintained for each version of each compiler. For the
Open\,MPI example presented in Section~\ref{sec:Module_files}, this means
that the user then only sees a single module file for Open\,MPI 1.7.3 instead
of three: the one providing the installation built using the compiler module
that is currently loaded:

{\small
\begin{alltt}
    \textbf{% module avail}
    ------------ <\emph{prefix}>/Core ------------
    GCC/4.8.2   Intel/14.0  Clang/3.4
    \textbf{% module load GCC/4.8.2}
    \textbf{% module avail}
    ----- <\emph{prefix}>/Core -----
    ------------ <\emph{prefix}>/Core ------------
    GCC/4.8.2   Intel/14.0  Clang/3.4
    ----- <\emph{prefix}>/Compiler/GCC/4.8.2 -----
    OpenMPI/1.7.3
\end{alltt}
}


Obviously, such a hierarchy is not limited to a single level. For example,
the module files for different MPI implementations can further extend
\texttt{\$MODULEPATH} to enable all modules which depend on both the loaded
compiler and the respective MPI installation.


\label{sec:hierarchical_advantages}
\subsection{Advantages over traditional module naming schemes}

\markus{This sub-subsection IMHO only makes sense if there would be a second
one. I propose to remove it and simply list the advantages as part of Section
III.A.}

A hierarchical module file organization has a number of significant
advantages. First, at any point in time, users are only presented with the
modules which are meaningful in the current context. That is, the list of
available modules is much shorter and therefore less overwhelming. Second,
encoding the dependency chain in the module name is no longer necessary for
modules enabling another hierarchy level, thereby leading to more intuitive
module names. And finally, loading of incompatible modules is automatically
avoided, preventing users from making simple mistakes which may lead to
subtle errors that are non-obvious and hard to debug.

\markus{Somewhere, we should mention that the hierarchical module scheme as
outlined above \textit{in principle} also works with C/Tcl or Tcl-only
modules, but that specific features simplify usage a lot: ``stack-based'' swap,
module deactivation, \texttt{module spider}, etc. But there is a small
chicken/egg problem here \ldots}

\remark{basic users make less mistakes, expert users retain full freedom}

\label{sec:hierarchical_consequences}
\subsection{Consequences of using a hierarchical naming scheme}

\remark{\itemize{
    \item complicates (manual) creation of module files due to hierarchical organization
    \item requires enhanced modules too to deal with idiocracies of efficiently using the provided modules (search, swapping between module stacks)
}}

\easybuild{} (section~\ref{sec:easybuild}) automates creation of module files under a specified naming scheme;

Lmod (section~\ref{sec:lmod}) provides required functionality to let users easily navigate and use modules in a hierarchical scheme
