In a typical desktop environment, it is usually sufficient to have a
single version of a software package installed to fulfill a particular purpose. HPC
systems, on the other hand, are normally used by a large user community with widely varying
demands. In particular, there is often the need to make multiple versions of
software packages available. These sometimes include competing packages
that provide either identical or significantly overlapping functionality. Examples include
 different implementations of the MPI standard (e.g., Open\,MPI vs.
MVAPICH) and linear algebra packages (e.g., OpenBLAS vs. Intel MKL).

A simple yet powerful solution to this issue are environment
modules~\cite{furlani91,furlani96,eadline,laytonEM1}, which allow
users to easily load, unload, and switch between software packages by
modifying the user's environment. This approach achieves this by adjusting environment
variables like \texttt{\small\$PATH} and/or setting additional package-specific
variables to specify a license server.
However, while environment modules are used by many
HPC sites around the world this approach is not without its challenges,
a common one being managing large numbers of modules in a way that allows individual
users to easily control their own environments.

Providing users an easy way to access the (scientific) software
available on an HPC system is certainly a challenge, but installing large collections
of scientific software packages is a
non-trivial task in its own right. These packages are often written by domain
scientists who are comfortable only with their their own hardware and software
environment; the developers are often not interested in providing a robust
build and installation procedure based on portable build
tools~\cite{Dubois03}. As a result, system administrators of different HPC sites may
find they have to reinvent the wheel to
get a particular software package installed on their local system. Moreover, this
kind of hard-won experience typically is not shared between HPC sites. In
addition, the required modifications and the exact installation steps are
often poorly---if at all---documented, which significantly impedes
maintainability and reproducibility of software installations.

In this paper, we address these issues by introducing an automated approach to
installing scientific software and organizing the corresponding modules in a
hierarchical way. This is achieved by leveraging the functionality provided by the
two community-driven tools \emph{\easybuild{}}~\cite{EasyBuildSC12} and
\emph{Lmod}~\cite{taccLmod}. Lmod provides a significantly enhanced but
(largely) backward-compatible implementation of environment modules including specific
features targeting a hierarchical module organization. EasyBuild provides a
framework for automating software installations with a particular focus on
scientific software packages---with the intention to collect and share the
knowledge that is currently distributed in the HPC community.

The remainder of this paper is structured as follows. In Section~\ref{sec:traditional},
we describe the traditional approach to installing scientific software stacks on HPC
systems, highlighting the common problems experienced by user support staff and end
users. Section~\ref{sec:hierarchical} proposes hierarchical module naming schemes as a 
promising alternative, and outlines the associated implications and issues. In
Sections~\ref{sec:easybuild} and \ref{sec:lmod}, we argue that \easybuild{} and Lmod
are well suited  for dealing with hierarchical module naming schemes, and describe how 
these tools enhance both the user experience and the efficiency of user support teams.
The community-oriented aspect of both tools, and the synergy between them, is the
subject of Section~\ref{sec:communities_synergy}. We discuss future and related work in
Sections~\ref{sec:future_work} and \ref{sec:related_work} respectively, and offer
concluding remarks in  Section~\ref{sec:conclusion}.

%\remark{issues with installing \& providing scientific software}
%\remark{importance of appropriate tools \& community}
