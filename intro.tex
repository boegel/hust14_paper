Unlike a typical desktop environment where it is usually sufficient to have a
single software package installed to fulfill a particular purpose, HPC
systems are normally used by a large user community with widely varying
demands. In particular, there is often the need to make multiple versions of
a software package available, and sometimes even conflicting packages
providing either identical or significantly overlapping functionality, such
as different implementations of the MPI standard (e.g., Open\,MPI vs.
MVAPICH) or linear algebra packages (e.g., OpenBLAS vs. Intel MKL).

A simple yet powerful solution to this issue are environment
modules~\cite{furlani91,furlani96,eadline,cmod,laytonEM1}, which allow
users to easily load, unload, and switch between software packages by
modifying the user's environment, that is, adjusting environment
variables such as \texttt{\$PATH} and \texttt{\$MANPATH} and/or setting
package-specific variables, for example, to specify the DNS name of a
license server. However, while environment modules are used by many
HPC sites around the world, dealing with the subtleties of different
implementations as well as organizing large numbers of modules that
get added over time such that users can easily set up their working environment
remains a major challenge.

In addition to providing users an easy way to access the software available
on the HPC system, installing scientific software packages is a non-trivial
task in its own right. As these packages are often written by domain
scientists with a strong focus on conducting their own research on platforms
they have access to, less emphasis is placed on portable build
systems~\cite{Dubois03}. That
is, system administrators of different HPC sites may reinvent the wheel to
get a particular software package installed on their local system, as this
kind of knowledge typically is not shared, especially between sites. In
addition, the required modifications and the exact installation steps are
often poorly---if at all---documented, which significantly impedes
maintainability and reproducibility.

In this paper, we introduce an automated approach to installing scientific
software and organizing the corresponding modules in a hierarchical way to
address the aforementioned shortcomings. This is achieved by advantageously
combining the functionality provided by the two community-driven tools
\emph{\easybuild{}}~\cite{EasyBuildSC12} and \emph{Lmod}~\cite{taccLmod}.
While Lmod provides a significantly enhanced but
(mostly) backward-compatible implementation of environment modules including specific
features targeting a hierarchical module organization, EasyBuild provides a
software build and installation framework with a particular focus on
scientific software packages---with the intention to collect and share the
knowledge that is currently distributed in the HPC community.

The remainder of this paper is structured as follows. First, we describe the
tradiotional way of installing scientific software stacks on HPC systems in
Section~\ref{sec:traditional}, highlighting the common problems that both user
support teams and end users run into. In Section~\ref{sec:hierarchical} we
present the use of a hierarchical module naming scheme as an alternative, and
explain how it can help in overcoming these problems. Subsequently, in
Sections~\ref{sec:easybuild} and \ref{sec:lmod}, we discuss how \easybuild{} and
Lmod are appropriate tools for effectively dealing with the highlighted issues, 
and how they can further enhance both the user experience and efficiency of user
support teams. We discuss future work in Section~\ref{sec:future_work} and related
work in Section~\ref{sec:related_work}. Finally, we conclude in
Section~\ref{sec:conclusion}.

%\remark{issues with installing \& providing scientific software}
%\remark{importance of appropriate tools \& community}
