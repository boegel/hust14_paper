Unlike a typical desktop environment where it is usually sufficient to have a
single software package installed to fulfill a particular purpose, HPC
systems are normally used by a large user community with widely varying
demands. In particular, there is often the need to make multiple versions of
a software package available, and sometimes even conflicting packages
providing either identical or significantly overlapping functionality, such
as different implementations of the MPI standard (e.g., Open\,MPI vs.
MVAPICH) or linear algebra packages (e.g., OpenBLAS vs. Intel MKL).

A simple yet powerful solution to this issue are environment
modules~\cite{environment_modules_paper}, which allow users to easily load,
unload, and switch between software packages by  modifying the user's
environment, that is, adjusting environment variables such as \texttt{PATH}
or \texttt{MANPATH} and/or setting package-specific variables, for example,
to define the DNS name of a license server. However, while environment
modules are used by many HPC sites around the world, dealing with the
subtleties of different implementations as well as organizing large numbers
of modules that get added over time remains a major challenge.

In addition to providing users an easy way to access the software available
on the HPC system, installing scientific software packages is a non-trivial
task in its own right. As these packages are often written by domain
scientists with a strong focus on conducting their own research on platforms
they have access to, less emphasis is placed on portable build systems.
\remark{some more text needed}

In this paper, we introduce an automated approach to installing scientific
software and organizing the corresponding modules in a hierarchical way to
address the aforementioned shortcomings. This is achieved by advantageously
combining the functionality provided by the two community-driven tools
EasyBuild and Lmod.

The remainder of this paper is structured as follows: \ldots

\remark{issues with installing \& providing scientific software}
\remark{importance of appropriate tools \& community}
