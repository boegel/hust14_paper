Lmod~\cite{laytonLmod,taccSecretSauce,taccLmod} is a Lua-based~\cite{LuaBook},
modern alternative to the traditional environment modules implementations.
It is a drop-in replacement for the Tcl/C
and Tcl-only implementations, except for a few corner cases.
It can consume module files written in both Tcl and
Lua, and has as its primary design goal empowering users
and improving the user experience without hindering experts.

Lmod began as a prototype designed to test ideas on how to manage a hierarchical
module system.  Early on, it became clear that existing tools were inadequate; 
the developers began a complete rewrite based on the clean
and powerful constructs of the Lua programming language.
Lua also fills the same niche that for which Tcl was
originally designed: both languages embed and extend easily into
other programs.  Lmod's pre-release versions proved to be
fast, flexible, and robust; the system quickly moved from
a prototype to a worthy replacement for the older, established modules tools.

October~2008 marked the first public release of Lmod; it has been under active
development ever since. TACC adopted Lmod for its production systems in October~2009.
Since February~2011, there have been 153 tagged versions and 30+ official releases.
At the time of writing, the latest available version is v5.7.2.

\subsection{Support for hierarchical module naming schemes}

The key to Lmod's support for a module hierarchy is the fact that it monitors
changes to \texttt{\small \$MODULEPATH}. Doing so enables it to support the
advanced notion of swapping hierarchical modules from the ground up, resolving all
the issues described in Sections~\ref{sec:using_a_hierarchy}.

Swapping one (core) compiler module for another using Lmod triggers
a chain of events.  The loaded compiler module is unloaded, 
causing the matching paths to be removed from \texttt{\small \$MODULEPATH}; 
 any loaded modules for MPI and other packages that depend on it
are unloaded as well, because the respective module files are no longer available
in the active module search path. These modules are marked \emph{inactive}.
Next, the new compiler module is loaded, causing new entries
to be added to the \texttt{\small \$MODULEPATH}. This triggers a search
for a compatible module in the new module search path for each inactive module.

For example, swapping a loaded \texttt{\small GCC} compiler module for the
\texttt{\small {Clang}} module results in reloading the modules that
depend on it, e.g., \texttt{\small FFTW} (a parallel FFT library) and
\texttt{\small MPICH} (an MPI library):
{\small
  \begin{alltt}
    \textbf{\% module list}
    Currently loaded modules:
    1) GCC/4.8.2  2) MPICH/3.1.1  3) FFTW/3.3.2
    \textbf{\% module swap GCC Clang}
    The following have been reloaded:
    1) FFTW/3.3.2  2) MPICH/3.1.1
    \textbf{\% module list}
    Currently loaded modules:
    1) Clang/3.4  2) MPICH/3.1.1  3) FFTW/3.3.2
\end{alltt}
}
\noindent
Lmod lists reloaded modules
alphabetically after a \texttt{\small module swap}, while the output of
the \texttt{\small module list} command lists them in the order in which
they were loaded.

Another key issue with a hierarchical layout of modules is that not all
existing modules are visible when a user executes \texttt{\small module avail} (see
Section~\ref{sec:hierarchical_consequences_visibility}). Lmod includes a
new command, \texttt{\small module spider}, to search for modules across the
entire module tree and report all \emph{existing} modules. The semantics of
\texttt{\small module avail} is the same as it is with other module tools to ensure
compatibility; this command reports all modules that can be \emph{loaded} in the
current context (determined by \texttt{\small \$MODULEPATH}).

The ability to maintain a consistent set of hierarchical modules and navigate the
module tree outside of the current \texttt{\small \$MODULEPATH} are among the key
strengths of Lmod. There are, however, a number of other interesting features, many
of which have been implemented in response to requests by the members of an
energetic, enthusiastic, and demanding Lmod community.

\subsection{The \texttt{\small ml} command and unload/swap shortcut}

Lmod provides a additional command named \texttt{\small ml} for those who tend to
misspell the \texttt{\small moduel}, \texttt{\small mdoule}, \emph{err}
\texttt{\small module} command, which focuses on
commonly used subcommands. Without arguments \texttt{\small ml} is a synonym for
\texttt{\small module list}, while using it with an argument that is not
recognized as a subcommand (e.g., \texttt{\small ml foo}) corresponds to loading
the specified module (e.g., \texttt{\small module load foo}).
\texttt{\small ml} accepts other subcommands associated with \texttt{\small module};
for example \texttt{\small ml avail} and \texttt{\small module avail} are equivalent.
In the unlikely case that a module name is the same as a subcommand supported by Lmod, 
some care must taken when using \texttt{\small ml}. For example, loading a module
named \texttt{\small spider} should be written as \texttt{\small ml load spider}.

Another shortcut supported by the Lmod command line is unloading modules by
prefixing module names with a minus sign (`\texttt{\small-}'). This also allows users to swap both the compiler and MPI stack with a single command, for example:
{\small
  \begin{alltt}
      \textbf{\% ml -GCC -MPICH Clang OpenMPI}\
  \end{alltt}
}
\noindent
which is equivalent to:{\small
  \begin{alltt}
      \textbf{\% module swap GCC Clang}
      \textbf{\% module swap MPICH OpenMPI}\
  \end{alltt}
}

\subsection{Properties}

Lmod also makes it possible to assign \emph{properties} to modules,
indicating that a package has some particular capability or
characteristic. This is particularly useful on modern supercomputers that include
one or more types of accelerators, e.g., GPUs or Intel Xeon Phi coprocessors.
To designate that libraries or applications support execution on (one of) these
accelerators, one can assign properties in the corresponding module files
using the \texttt{\small add\_property} function. 
Module properties are reported in the output of \texttt{\small module} subcommands,
making them visible to users:
{\small
\begin{alltt}
  \textbf{\% module list}
  Currently Loaded Modules:
    1) Intel/14.0.2        3) MPICH/3.1.2
    2) imkl/11.1.3.174 \textbf{(*)} 4) Boost/1.55.0 \textbf{(P)}
  Where:
   \textbf{(*)}:  supports host, native and offload
   \textbf{(P)}:  built for host and native Phi
\end{alltt}
}
\noindent Another use of this feature is marking modules as `alpha'
or `beta' to characterize the maturity and stability of the 
associated software. Sites can easily add properties of their own.

\subsection{Caching}

Support for module properties does add a complication to the module
system.  Without properties, the \texttt{\small avail} subcommand only requires the
names of available module files, not their contents.  To support properties, Lmod
must parse the contents of all module files, whether or not these files include
properties. On some file systems this can be slow.  To mitigate this, Lmod supports
\emph{caching} of module files, as reading one single (possibly large)
file is faster than walking a directory tree and reading lots of small files.
Caching also makes other subcommands, e.g., \texttt{\small module spider},
significantly faster. An Lmod module file cache is also referred to as
`spider cache'; both system-level and user-level caches are supported.

Lmod contains hooks to check whether a cache is up-to-date and will automatically
update a user-level cache when deemed necessary, i.e., when executing a particular
subcommand takes too long. System cache files can be recomputed after (re)installing 
software and/or periodically via a dedicated cron job. Usually, the presence of a
system-level cache eliminates the need for a user-level cache. Lmod trusts existing
cache files in all cases except for the \texttt{\small load} subcommand, so
modules can be loaded even if the cache is slightly out-of-date,
e.g., when installing a stack of software.

%\kenneth{also mention system vs user cache?}
%\robert{I don't know}

\subsection{Module families}

Another Lmod enhancement is support for module
\emph{families} via calls to the \texttt{\small family} function in module files.
Families are an alternative approach to managing module conflicts; only one module of
a particular family can be loaded at any given time. While conflicts can only be
defined based on module names, module family `labels` can be chosen freely and
are easy to maintain. A new family member does not require changes to existing
module files.

For example, one might define a \texttt{\small compiler} family, by
simply including \texttt{\small family("compiler")} in each compiler module file.
This would result in the following meaningful error message when a user tries to
load two compiler modules at the same time:

{\small
\begin{alltt}
  \textbf{\% module load Clang}
  \textbf{\% module load GCC}
  Lmod has detected the following error:
  You can only have one compiler module loaded.
  To correct this, please do the following:
      module swap Clang GCC\
\end{alltt}}
\noindent
The constraint of only allowing to load one module per family can be
disabled by using Lmod in `expert mode', by defining the
\texttt{\small \$LMOD\_EXPERT} environment variable.

\subsection{Customizing Lmod behavior and hooks}

Another key Lmod feature is support for site-specific 
customizations via
a file named \texttt{\small SitePackage.lua}. Lmod provides several
\emph{hook} functions, allowing sites to plug in additional functionality.
For example, using the available hook for the \texttt{\small load} subcommand, a
site can add a new accompanying action. A typical example is a message logger: by
adding this action to the \texttt{\small load} subcommand, a site can collect and
analyze module usage across the entire system.

\ignore{
\subsection{Sensible version ordering}

There have been many improvement to improve usability for both users
and staff.  One of these is sensible version ordering.  Originally all
module systems use lexicographical ordering.  This means that version
9.0 is considered newer than 10.0.  Now Lmod knows that version 10.0 $>$
9.7.1 $>$ 9.7 $>$ 9.7rc1 $>$ 9.7a2 and so on.  It knows about
numerical order in any of the digits so 1.10.12 $>$ 1.10.9 $>$ 1.9.10 $>$ 1.9.2
and so on.

%\kenneth{mention version since when Lmod support sensible version ordering, and maybe flesh out the example a bit more? you can use EasyBuild v1.9.10 vs v1.10.0 as an example ;-)}
}

\subsection{\texttt{\small pushenv}: stack-based setting of environment variables}
\label{sec:lmod_pushenv}

Environment module systems support the \texttt{\small setenv} function for 
defining environment variables via module files. When a module file is unloaded,
any environment variables that were defined by that module are cleared. In some
cases, however, this can be problematic: the environment variable may have had a
meaningful value before the module was loaded.

Lmod enables a more sophisticated approach by supporting a stack-based
alternative via the \texttt{\small pushenv} function. For environment
variables defined in this way, Lmod maintains a hidden stack in the environment
so that previous values can be recovered.
For example, consider a module for the GCC compiler that
defines \texttt{\small \$CC} to be \texttt{\small gcc}, while a module for the
Open\,MPI library sets \texttt{\small \$CC} to
\texttt{\small mpicc}. The following table shows the value of \texttt{\small \$CC}
depending on whether \texttt{\small setenv} or \texttt{\small pushenv} is used:
\begin{center}
 \begin{tabular}{l|l|l}
 command                                 & \texttt{\small setenv} & \texttt{\small pushenv}\\
 \hline
 \textbf{\texttt{\small export CC=icc}}  & \texttt{\small icc}  & \texttt{\small icc} \\
 \textbf{\texttt{\small module load   GCC}}   & \texttt{\small gcc}    & \texttt{\small gcc}  \\
 \textbf{\texttt{\small module load   OpenMPI}} & \texttt{\small mpicc}  & \texttt{\small mpicc} \\
 \textbf{\texttt{\small module unload OpenMPI}} & \emph{(none)}   & \texttt{\small gcc}  \\
 \textbf{\texttt{\small module unload GCC}}   & \emph{(none)}   & \texttt{\small icc} \\
  \hline
    \end{tabular}
\end{center}
\noindent
Note that modules using \texttt{\small pushenv} properly restore the previous value
for \texttt{\small\$CC} when they are unloaded.

%\kenneth{include an example of the effect of using \texttt{pushenv} rather than \texttt{setenv}, maybe using \texttt{\$CC}?}

\subsection{Module collections}

One final feature of Lmod worth mentioning is the support for user-defined
\emph{collections} of modules. Collections allow users to define and name sets of
module files that can be loaded with a single command, or automatically on startup.
Users need only load the desired modules and then execute
\texttt{\small module save} with an optional name specified as an argument;
omitting the name defines the default collection. Users can list the available
collections via \texttt{\small module savelist} or the shorthand
\texttt{\small ml sl}. A collection can be restored by specifying its name to the
\texttt{\small restore} subcommand. The \texttt{\small module restore} command
without an argument (or starting a  new login session) reloads the default module
collection. This is a powerful, popular way to efficiently manage working
environments.

%, and allows user support teams to define a default
%set of loaded modules. That way, they can avoid that (novice) users
%are forced to pick a particular compiler and MPI module under a hierarchical module
%naming scheme before they can get started.

%\kenneth{try and give a rough idea of the size of the Lmod community?}

%\subsection{Key features}
%
%\begin{itemize}
%    \item user-friendly
%    \begin{itemize}
%        \item ml
%        \item sensible version ordering
%        \item case-insensitive module avail
%        \item module load => swap if needed
%        \item recursive unload (is default)
%        \item spider cache => fast avail (\& spider)
%    \end{itemize}
%    \item hooks
%    \item path priorities
%    \item properties
%    \item families
%    \item load with version range
%    \item pushenv
%\end{itemize}
%\remark{too long a list to include all? AP Robert: which ones do we really need to mention?}
%\remark{module swap works}
%\remark{module avail vs module spider}
%\remark{(spider cache)}
