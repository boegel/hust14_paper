Lmod~\cite{laytonLmod,taccSecretSauce,taccLmod} is a modern
replacement for environment modules.  With rare exceptions it is a
drop in replacement for Tcl/C modules. It can consume modulefiles
written in Tcl as well as Lua\cite{LuaBook}.  It has strong focus on
providing users easy access to their software stack, without hindering
experts.  

The original idea for Lmod was to prototype a module system to test
ideas about how to handle a hierarchical module system.  It was clear
at the beginning that it was going to require a complete rewrite of
the system, so the Lua language was choosen because of clean yet
powerful constructs.  Lua also fills the same niche that Tcl was
originally designed for, namely, it embeds and extends easily into
other programs.  As time went on, it was clear that the Lmod modules
system written in Lua implementation was fast enough to use directly.  

%\kenneth{saying it's "very fast" and then saying it's "fast enough" feels
%contradictory}

The key idea that Lmod exploits is that it watches changes to
\texttt{\$MODULEPATH}.  In a hierarchical layout, a compiler modulefile adds to
MODULEPATH a new directory of compatible modules.  Similarly an MPI
implementation's modulefile will add a directory for the compatible
parallel libraries and parallel applications that work with that
compiler and MPI pairing.  When swapping one compiler for another will
cause a chain of events.  First the old compiler module is unloaded
which causes the mpi and parallel packages to be unloaded because the
modulefiles can no-longer be found.  These modules are marked
inactive.

Second step is to load the new compiler.  This means that a new entry
is added to MODULEPATH and any inactive modules will search the new
directory for compatible modules as show below:
{\small
  \begin{alltt}
    \% module swap gcc clang

    The following have been reloaded:
    1) petsc/3.4  2) mpich/3.1.1
      
\end{alltt}
}
\noindent
In this case swapping the compilers swapped out a parallel solver
library, \emph{petsc} and the mpi stack, \emph{mpich}. This ability
to maintain a consistent set of modules when coupled with a
hierarchical module layout is Lmod's key strength but there are many
more features.  Many of these have been in response to community
requests.  

One of the first problems that shows up with a hierarchical module
layout, is that you can't see all the modules with avail.
Unfortunately, in the module command context.  ``\texttt{module avail}''
reports all the modules that you can load right now.  Lmod added a new
command ``\texttt{module spider}'' to walk the whole module tree and
report every possible module.


The module system is a key interface between users and system
administrators.  
%For sites that use Modules systems,  it is how users
%access the software that they need to get their work done.   
As the computer systems change so too do the requirements on the module
system.  One of these adding properties to modules.  Many
supercomputers now have one or more accelerators currently either GPU's
or XEON Phi's also known as MIC's.   In particular, XEON Phi's are a
daughter card across the PCI bus and to create applications to run on
the XEON Phi's requires cross compilation.  This means that libraries
are needed for the Phi's.  Due to the way the environment variables
are handled, a single modulefile can serve regular and Phi libraries.

But user's won't know which libraries are Phi-aware.  Lmod added
functions so that properties could be added to modulefiles.  This way
when modules are listed via module list, and avail the properties can
be reported.  This is just one use of properties.  It is under site
control what properties are reported.  For example, some sites use it
to mark alpha and beta releases of software packages.

Added support for properties added a complication to the module
system.  Without properties, the avail command just needed the name of
the module not the contents.  With properties, the contents of every
module needed to be read and evaluated, whether the modulefile had
properties or not. On parallel filesystems such as lustre this can be
slow.  To mitigate this, Lmod added the ability to have cache files.
It turns out that reading one large file is much faster than walking a
directory tree and reading several small files.  There is support for
both user and system cache files.  Keeping the cache files current can
be a challenge when installed software is being updated.  One approach
is to have the installing software system recomputes the cache after
completing the system update or the cache file is updated frequently
and hooks are in-place to know when the cache file is valid or not.

One of upshots of this is that Lmod trusts the cache files for
everything except for loading a modulefile.  This way a module can be
loaded even if the cache is out-of-date.  This can happen while
installing software (e.g. for EasyBuild) but users will see an updated
cache file.

Lmod is actively maintained and responsive to its community.  Since
February of 2011, there have been 153 tagged version and 30+ official
releases of Lmod.  Many of the improvements to Lmod have come from
requests from the user community.  

One of the improvement to Lmod is that there is the family function.
The way this works is that a compiler or mpi stack can be marked to be
part of a family and you only get one member of the family loaded at
one time.  It is possible to override this for the few expert user who
wish to have two compilers loaded at the same time.

Another key feature is the ability for sites to create a
\texttt{SitePackage.lua} file to modify the behavior of Lmod to suit
their needs.  Lmod provides ``hook'' functions that site can add
functionality.  There is a load hook.  This means that
every time a module is loaded a call to syslog can be
made to record a modules use.  Data from syslog can be collected to
know what modules are being used or not.

There have been many improvement to improve usability for both users
and staff.  One of these is sensible version ordering.  Originally all
module systems use lexicological ordering.  This means that version
9.0 is considered newer than 10.0.  Now Lmod knows that version 10.0 $>$
9.7.1 $>$ 9.7 $>$ 9.7rc1 $>$ 9.7a2 and so on.

Among the many feature that the Easybuild community have requested,
two stand out.  The first one ``\texttt{pushenv}''.  The module system
has supported \texttt{setenv} function which sets an environment variable 
upon load a module and clears its value upon unload.  There are times
when users and administrators would like the previous value recovered
rather than cleared.  This is what \texttt{pushenv} does.  It
maintains a hidden stack in the environment so it can recover old
values.

The second feature was an speed improvement.  Internally EasyBuild
builds the software package and creates the module file.  Then it
checks to see if it is available and loads and unloads it.  Lmod has
to walk the tree and read the modules to know what properties are
there. However EB is performing a ``module --terse avail'' is provides
an easily parsed output.  Since this output does not provide any
module properties the modulefiles do not need to be read.  Tests done
show that cuts the testing time significantly.

%\subsection{Key features}
%
%\begin{itemize}
%    \item user-friendly
%    \begin{itemize}
%        \item ml
%        \item sensible version ordering
%        \item case-insensitive module avail
%        \item module load => swap if needed
%        \item recursive unload (is default)
%        \item spider cache => fast avail (\& spider)
%    \end{itemize}
%    \item hooks
%    \item path priorities
%    \item properties
%    \item families
%    \item load with version range
%    \item pushenv
%\end{itemize}
%\remark{too long a list to include all? AP Robert: which ones do we really need to mention?}
%\remark{module swap works}
%\remark{module avail vs module spider}
%\remark{(spider cache)}
