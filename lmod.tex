Lmod~\cite{laytonLmod,taccSecretSauce,taccLmod} is a modern alternative to
the traditional environment modules implementations, and is mostly written in
Lua~\cite{LuaBook}. With rare exceptions, it is a drop-in replacement for the Tcl/C
and Tcl-only implementations. It can consume modulefiles written in Tcl as well as
Lua, and has strong focus on providing users easy access to their software stack,
without hindering experts.

The original idea for Lmod was to prototype a module system to test
ideas about how to handle a hierarchical module system.  Early on, it was clear
to the developers that the design of existing tools was inadequate, and that this
was going to require a complete rewrite, so the Lua language was chosen because
of its clean yet powerful constructs.  Lua also fills the same niche that Tcl was
originally designed for, namely, it embeds and extends easily into
other programs.  As time went on, it was clear that the version of Lmod written
in Lua was fast enough to be used directly, and hence the project
continues to use Lua for further development.

Lmod was first publicly released in Oct.~2008, and has been under active development
ever since. Since February of 2011, there have been 153 tagged version and
30+ official releases of Lmod. The latest available version to date is Lmod v5.7.3.
Since Oct.~2009, it is the production modules tool on HPC systems at TACC.

\subsection{Support for hierarchical module naming schemes}

The key idea that Lmod exploits is that it watches changes to
\texttt{\small \$MODULEPATH}. As such, it supports the advanced notion of
swapping modules in a hierarchical layout from the ground up, and thus
deals with the issues described in
Sections~\ref{sec:hierarchical_consequences_extensions} and
\ref{sec:hierarchical_consequences_availability}.

Swapping one (core) compiler module for another using Lmod will trigger
a chain of events.  The loaded compiler module will be unloaded, 
causing the matching paths to be removed from the \texttt{\small \$MODULEPATH},
and any loaded modules for MPI and parallel packages that depend on it
to be unloaded as well, because the respective module files are no longer available
in the active module search path. These modules are marked \emph{inactive}.
Next, the new compiler module is loaded which causes a new entry
to be added to the \texttt{\small \$MODULEPATH}, triggering a search
for a compatible module in the new module search path for each of the
modules that were marked inactive.

For example, swapping a loaded \texttt{\small gcc} compiler module for the
\texttt{\small {clang}} module results in reloading the loaded modules that
depend on it, e.g., \texttt{\small fftw3} (a parallel FFT library) and
\texttt{\small mpich} (an MPI library):
{\small
  \begin{alltt}
    \textbf{\% module list}
    Currently loaded modules:
    1) gcc/4.8.2  2) mpich/3.1.1  3) fftw3/3.3.2
    \textbf{\% module swap gcc clang}
    The following have been reloaded:
    1) fftw3/3.3.2  2) mpich/3.1.1
    \textbf{\% module list}
    Currently loaded modules:
    1) clang/3.4  2) mpich/3.1.1  3) fftw3/3.3.2
\end{alltt}
}
\noindent
The reloaded modules reloaded are printed out
alphabetically by the \texttt{\small module swap} command, while the output of
the \texttt{\small module list} command lists the modules in the order in which
they were loaded.

% \kenneth{does the order of the reloaded modules make sense? shouldn't
%   MPICH be reloaded first, since PETSc depends on it?}

Another key issue with a hierarchical layout of modules is that not all
existing modules are visible via \texttt{\small module avail} (see
Section~\ref{sec:hierarchical_consequences_visibility}). Lmod defines a
new command, \texttt{\small module spider}, to search for modules across the
entire module tree and report all \emph{existing} modules. The semantics of
the \texttt{\small module avail} command, i.e., reporting all modules that can
be \emph{loaded} in the current context (determined by
\texttt{\small \$MODULEPATH}) is retained, for compatibility with other module
tools.

The ability to maintain a consistent set of modules when coupled with a
hierarchical module layout, and navigating the module tree outside of the
current \texttt{\small \$MODULEPATH} is the key strength of Lmod, but there are
many more features.  Many of these have been implemented in response to
requests by the members of the Lmod community.

\subsection{\texttt{\small ml} command and unload/swap shortcut}

Lmod provides a short-hand command named \texttt{\small ml} for those who tend to
misspell the \texttt{\small moduel}, \texttt{\small mdoule}, \emph{err}
\texttt{\small module} command, which focuses on
commonly used subcommands. Without arguments \texttt{\small ml} is a synonym for
\texttt{\small module list}, while using it with an argument that is not
recognized as a subcommand (e.g., \texttt{\small ml foo}) corresponds to loading
the specified module (e.g., \texttt{\small module load foo}).
\texttt{\small ml} accepts other subcommands just like \texttt{\small module} does,
i.e., \texttt{\small ml avail} and \texttt{\small module avail} are equivalent.
In the unlikely case that a module name is the same as subcommand supported by Lmod, 
some care must taken when using \texttt{\small ml}. For example, loading a module
named \texttt{\small spider} should be written as \texttt{\small ml load spider}.

Another shortcut supported by the Lmod command line is to unload modules by
prefixing module names with a minus sign (`\texttt{\small-}'). This allows to for
example swap both the compiler and MPI stack with a single command, e.g.:
{\small
  \begin{alltt}
      \textbf{\% ml -gcc -mpich clang openmpi}\
  \end{alltt}
}
\noindent
is equivalent to:
{\small
  \begin{alltt}
      \textbf{\% module swap gcc clang}
      \textbf{\% module swap mpich openmpi}\
  \end{alltt}
}

\subsection{Properties}

Lmod supports the notion of assigning \emph{properties} to modules,
indicating that the software they provide has some particular capability or
characteristic. This is particularly useful on modern supercomputers that include
one or more types of accelerators, e.g., GPUs or Intel Xeon Phi coprocessors.
To indicate whether libraries and applications support execution on (one of) these
accelerators, properties can assigned in the corresponding module files
using the \texttt{\small add\_property} function. Supported module
properties are user-definable.

Module properties are reported in the output of \texttt{\small module} subcommands,
hence users can tell which software is for example Phi-aware:
{\small
\begin{alltt}
  \textbf{\% module list}
  Currently Loaded Modules:
    1) intel/14.0.2       3) mpich/3.1.2
    2) mkl/mkl      \textbf{(*)}   4) boost/1.55.0 \textbf{(P)}

  Where:
   \textbf{(*)}:  supports host, native and offload
   \textbf{(P)}:  built for host and native Phi
\end{alltt}
}
\noindent Another possible application of properties is to mark modules as `alpha'
or `beta', depending on the amount of testing the provided software builds have
received.

\subsection{Caching}

Support for module properties adds a complication to the module
system.  Without properties, the \texttt{\small avail} subcommand only requires the
names of available module files, not their contents.  However, with support for
properties the contents of every module file need to be read and evaluated,
regardless of whether the module file includes any properties or not. On parallel
file systems this can be terribly slow.  To mitigate this Lmod enables
\emph{caching} of module files, since reading one single (large) file is
significantly faster than walking a directory tree and reading several small files.
Hence, caching also makes other subcommands, e.g., \texttt{\small module spider},
significantly faster. An Lmod module file cache is also referred to as
`spider cache', and both system-level and user-level caches are supported.

Lmod contains hooks to check whether a cache is up-to-date and will automatically
update a user-level cache when deemed necessary, i.e., when executing a particular
subcommand takes too long. System cache files can be recomputed after (re)installing software and/or periodically via a dedicated cron job. Usually, the presence of a
system-level cache eliminates the need for a user-level cache.
Lmod trusts existing cache files for everything except loading a module
file, so modules can still be loaded even if the cache is out-of-date, e.g., when
installing a stack of software using \easybuild{}.

%\kenneth{also mention system vs user cache?}
%\robert{I don't know}

\subsection{Module families}

Another enhancement provided by Lmod is the support for defining module
\emph{families}, using the \texttt{\small family} function in (Lua) module files.
Families are an alternative to module conflicts, in the sense that only one module of
a particular family can be loaded at any given time. Other than conflicts, which
are based on module names, module family `labels' can be chosen freely and are easy
to maintain since they do not require to update existing module files when adding a
family member.

For example, a module family named \texttt{\small compiler} can be defined, by
simply including \texttt{\small family("compiler")} in each compiler module file.
This would result in the following meaningful error message when a user tries to
load two compiler modules at the same time:

{\small
\begin{alltt}
  \textbf{\% module load intel}
  \textbf{\% module load gcc}
  Lmod has detected the following error:
  You can only have one compiler module loaded.
  To correct this, please do the following:
      module swap intel gcc/4.8.1\
\end{alltt}}
\noindent
The constraint of only allowing to load one module per family can be
disabled by using Lmod in `expert mode', by defining the
\texttt{\small \$LMOD\_EXPERT} environment variable.

\subsection{Customizing Lmod behavior and hooks}

Another key feature is the ability for sites to customize the behavior of Lmod
by creating a file named \texttt{\small SitePackage.lua}. Lmod provides several
\emph{hook} functions, allowing sites to plug in additional functionality.
For example, using the available hook for the \texttt{\small load} subcommand, an
accompanying action can be assigned. That way, a message can be logged for every
module being loaded for example, allowing to figure out which modules are being
used or not. 

\ignore{
\subsection{Sensible version ordering}

There have been many improvement to improve usability for both users
and staff.  One of these is sensible version ordering.  Originally all
module systems use lexicographical ordering.  This means that version
9.0 is considered newer than 10.0.  Now Lmod knows that version 10.0 $>$
9.7.1 $>$ 9.7 $>$ 9.7rc1 $>$ 9.7a2 and so on.  It knows about
numerical order in any of the digits so 1.10.12 $>$ 1.10.9 $>$ 1.9.10 $>$ 1.9.2
and so on.

%\kenneth{mention version since when Lmod support sensible version ordering, and maybe flesh out the example a bit more? you can use EasyBuild v1.9.10 vs v1.10.0 as an example ;-)}
}

\subsection{\texttt{\small pushenv}: stack-based setting of environment variables}
\label{sec:lmod_pushenv}

Environment module systems support the \texttt{\small setenv} function for 
defining environment variables via module files. When a module file is unloaded,
any environment variables that were defined by that module are cleared regardless
of whether or not they were defined prior to the module being loaded (since the
previous value is no longer available). For common environment variables that may
be defined in multiple module files that can be loaded at the same time, simply
clearing variables when modules are unloaded is not desirable.

Therefore, Lmod supports an alternative function for setting environment variables
stack-based, using the \texttt{\small pushenv} function. For environment
variables being defined that way, a hidden stack is maintained in the environment
so that previous values can be recovered.
For example, consider a module for the GCC compiler named \texttt{\small gcc} which
defines \texttt{\small \$CC} to be \texttt{\small gcc}, while a module for the
Open\,MPI library named \texttt{\small openmpi} sets \texttt{\small \$CC} to
\texttt{\small mpicc}. The following table shows the value of \texttt{\small \$CC}
depending on whether \texttt{\small setenv} or \texttt{\small pushenv} is used:
\begin{center}
 \begin{tabular}{l|l|l}
 command                                 & \texttt{\small setenv} & \texttt{\small pushenv}\\
 \hline
 \textbf{\texttt{\small export CC=icc}}  & \texttt{\small icc}  & \texttt{\small icc} \\
 \textbf{\texttt{\small module load   gcc}}   & \texttt{\small gcc}    & \texttt{\small gcc}  \\
 \textbf{\texttt{\small module load   openmpi}} & \texttt{\small mpicc}  & \texttt{\small mpicc} \\
 \textbf{\texttt{\small module unload openmpi}} & \emph{(none)}   & \texttt{\small gcc}  \\
 \textbf{\texttt{\small module unload gcc}}   & \emph{(none)}   & \texttt{\small icc} \\
  \hline
    \end{tabular}
\end{center}
\noindent
While \texttt{\small\$CC} is unset after unloading the \texttt{\small openmpi} module
using \texttt{\small setenv}, the modules using \texttt{\small pushenv} result
in properly restoring the previous value for \texttt{\small\$CC} on unload.

%\kenneth{include an example of the effect of using \texttt{pushenv} rather than \texttt{setenv}, maybe using \texttt{\$CC}?}

\subsection{Module collections}

One final feature of Lmod we would like to mention is the support for user-defined
\emph{collections} of modules. Collections allow users to define and name their
own sets of module files that can be loaded with a single command. Once the desired
set of modules is loaded, the \texttt{\small save} subcommand can be used
to define a module collection with the name specified as an argument; omitting an
argument defines the default collection. The list of available collections
can obtained via \texttt{\small module savelist} or the shorthand
\texttt{\small ml sl}. A collection can be restored by specifying its name to the
\texttt{\small restore} subcommand. Reloading the default module collection can be
done by simply using \texttt{\small module restore} without argument.
This powerful feature provides users with a way to efficiently manage their
different working environments.

%, and allows user support teams to define a default
%set of loaded modules. That way, they can avoid that (novice) users
%are forced to pick a particular compiler and MPI module under a hierarchical module
%naming scheme before they can get started.

%\kenneth{try and give a rough idea of the size of the Lmod community?}

%\subsection{Key features}
%
%\begin{itemize}
%    \item user-friendly
%    \begin{itemize}
%        \item ml
%        \item sensible version ordering
%        \item case-insensitive module avail
%        \item module load => swap if needed
%        \item recursive unload (is default)
%        \item spider cache => fast avail (\& spider)
%    \end{itemize}
%    \item hooks
%    \item path priorities
%    \item properties
%    \item families
%    \item load with version range
%    \item pushenv
%\end{itemize}
%\remark{too long a list to include all? AP Robert: which ones do we really need to mention?}
%\remark{module swap works}
%\remark{module avail vs module spider}
%\remark{(spider cache)}
