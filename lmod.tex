
Lmod is a modern tool for consuming module files, with a strong focus on providing users
easy access to their (scientific) software stack, without hindering experts. It is the
handshake between the system administrators and end users, and delivers a powerful yet
flexible way of configuring and managing their working software stack. Lmod is feature-rich,
well supported, continously enhanced, and comes with a vibrant
community.   This talk will cover the main concepts in a module
system, the choice of module layout (hierarchical vs. flat),  The
new features of Lmod and and how to leverage the module system to
track how which software package user use and do not use.

\remark{actively maintained}

\subsection{Key features}

\begin{itemize}
    \item user-friendly
    \begin{itemize}
        \item ml
        \item sensible version ordering
        \item case-insensitive module avail
        \item list/avail (can) go(es) to stdout
        \item module load => swap if needed
        \item recursive unload (is default)
        \item spider cache => fast avail (\& spider)
    \end{itemize}
    \item hooks
    \item path priorities
    \item properties
    \item families
    \item load with version range
    \item pushenv
\end{itemize}
\remark{too long a list to include all? AP Robert: which ones do we really need to mention?}
\remark{module swap works}
\remark{module avail vs module spider}
\remark{(spider cache)}
