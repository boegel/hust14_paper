\documentclass[conference, compsocconf]{IEEEtran}

%\usepackage[ruled]{algorithm2e}
%\renewcommand{\algorithmcfname}{ALGORITHM}
%\SetAlFnt{\small}
%\SetAlCapFnt{\small}
%\SetAlCapNameFnt{\small}
%\SetAlCapHSkip{0pt}
%\IncMargin{-\parindent}


\usepackage[english]{babel}
\usepackage{epsfig}
\usepackage{amsmath}
\usepackage{textcomp}
%\usepackage{multirow}
\usepackage{graphicx}
\usepackage{booktabs}
\usepackage[tight,footnotesize]{subfig}
\usepackage{enumerate}
\usepackage[hyphens]{url}
\usepackage{hyperref}
\usepackage{alltt}
\usepackage{xcolor}
\usepackage{cite}
\newcommand\Small{\fontsize{8.3}{8.6}\selectfont}
\usepackage{listings}
\lstset{
 tabsize=4,
        basicstyle=\ttfamily,
        %upquote=false,
        aboveskip=\baselineskip,
        belowskip=\baselineskip,
        columns=fixed,
        showstringspaces=false,
        extendedchars=true,
        breaklines=true,
        prebreak = \raisebox{0ex}[0ex][0ex]{\ensuremath{\hookleftarrow}},
        frame=lines,
        showtabs=false,
        showspaces=false,
        identifierstyle=\ttfamily,
        keywordstyle=\color[rgb]{0,0,1},
        commentstyle=\color[rgb]{0.133,0.545,0.133},
        stringstyle=\color[rgb]{0.627,0.126,0.941},
        language=Python
}


\newcommand{\ignore}[1]{}
\newcommand{\markus}[1]{\ifremark\textcolor{cyan}{\bf Markus: #1}\fi}
\newcommand{\kenneth}[1]{\ifremark\textcolor{magenta}{\bf Kenneth: #1}\fi}
\newcommand{\robert}[1]{\ifremark\textcolor{blue}{\bf Robert: #1}\fi}
%\newcommand{\markus}[1]{}
%\newcommand{\kenneth}[1]{}
%\newcommand{\robert}[1]{}

\newif\ifremark
\long\def\remark#1{
\ifremark%
        \begingroup%
        \dimen0=\columnwidth
        \advance\dimen0 by -1in%
        \setbox0=\hbox{\parbox[b]{\dimen0}{\protect\em #1}}
        \dimen1=\ht0\advance\dimen1 by 2pt%
        \dimen2=\dp0\advance\dimen2 by 2pt%
        \vskip 0.25pt%
        \hbox to \columnwidth{%
                \vrule height\dimen1 width 3pt depth\dimen2%
                \hss\copy0\hss%
                \vrule height\dimen1 width 3pt depth\dimen2%
        }%
        \endgroup%
\fi}

\remarktrue
%\remarkfalse

\newif\ifpaper
\long\def\paper#1{
\ifpaper%
    \begingroup%
    {#1}
    \endgroup%
\fi
}

\papertrue
%\paperfalse

\newif\iftechreport
\long\def\techreport#1{
\iftechreport%
    \begingroup%
    {#1}
    \endgroup%
\fi
}

%\techreporttrue
\techreportfalse

\renewcommand{\baselinestretch}{0.90} %837} %0.8633}


\newcommand{\easybuild}{\mbox{EasyBuild}}

\begin{document}


\title{A Modern Approach to Installing Scientific Software Using EasyBuild and Lmod}

\author{
    \IEEEauthorblockN{Markus Geimer}
    \IEEEauthorblockA{
        J\"ulich Supercomputing Centre (JSC)\\
        Forschungszentrum J\"ulich GmbH\\
        52425 J\"ulich, Germany\\
        m.geimer@fz-juelich.de
    }
\and
    \IEEEauthorblockN{Kenneth Hoste}
    \IEEEauthorblockA{
        HPC-UGent, DICT\\
        Ghent University\\
        Krijgslaan 281, S9\\
        B-9000 Gent, Belgium\\
        kenneth.hoste@ugent.be
    }
\and
    \IEEEauthorblockN{Robert McLay}
    \IEEEauthorblockA{
        Texas Advanced Computing Center (TACC)\\
        University of Texas\\
        10100 Burnet Rd\\
        Austin (TX) 78758, USA\\
        mclay@tacc.utexas.edu
    }
}



\maketitle


\remark{
desired takebacks:
\itemize{
    \item available manpower to provide end-user software is limited
    \item hierarchical module naming scheme is the way to go
    \item avoid manual software installation and creation of module files (tedious) with EasyBuild
    \item required Lmod to allow users to easily 'navigate'
    \itemize{
        \item module spider for looking beyond the current \$MODULEPATH
        \item module swap + deactivation of modules
    }
    \item synergy between EasyBuild \& Lmod
    \item community-driven development of both tools
}
}

\begin{abstract}
\remark{limit is 1000 characters, now at 986}

HPC user support teams invest a lot of time and effort in installing scientific
software for their users. A well-established practice is providing module files,
with the intention to make it easy for users to set up their working environment.
Several problems remain however: user support teams lack appropriate tools to maintain a
scientific software stack consistently, users still struggle to set up their working
environment correctly, etc.
In this paper we present a modern approach to installing scientific software, which provides
a solution to these commonly problems. We show how EasyBuild, a software build and
installation framework, can be used to fully automate the installation of scientific software.
By using a hierarchical module naming scheme to offer module files to users in a more structured
way and providing Lmod, a modern tool for working with modules, we avoid many of the common
mistakes made by users while retaining the flexibility required by power users. 

\end{abstract}

% For peer review papers, you can put extra information on the cover
% page as needed:
% \ifCLASSOPTIONpeerreview
% \begin{center} \bfseries EDICS Category: 3-BBND \end{center}
% \fi
%
% For peerreview papers, this IEEEtran command inserts a page break and
% creates the second title. It will be ignored for other modes.
\IEEEpeerreviewmaketitle



\section{Introduction}\label{sec:intro}
Unlike a typical desktop environment where it is usually sufficient to have a
single software package installed to fulfill a particular purpose, HPC
systems are normally used by a large user community with widely varying
demands. In particular, there is often the need to make multiple versions of
a software package available, and sometimes even conflicting packages
providing either identical or significantly overlapping functionality, such
as different implementations of the MPI standard (e.g., Open\,MPI vs.
MVAPICH) or linear algebra packages (e.g., OpenBLAS vs. Intel MKL).

A simple yet powerful solution to this issue are environment
modules~\cite{environment_modules_paper}, which allow users to easily load,
unload, and switch between software packages by  modifying the user's
environment, that is, adjusting environment variables such as \texttt{PATH}
or \texttt{MANPATH} and/or setting package-specific variables, for example,
to define the DNS name of a license server. However, while environment
modules are used by many HPC sites around the world, dealing with the
subtleties of different implementations as well as organizing large numbers
of modules that get added over time remains a major challenge.

In addition to providing users an easy way to access the software available
on the HPC system, installing scientific software packages is a non-trivial
task in its own right. As these packages are often written by domain
scientists with a strong focus on conducting their own research on platforms
they have access to, less emphasis is placed on portable build systems.
\remark{some more text needed}

In this paper, we introduce an automated approach to installing scientific
software and organizing the corresponding modules in a hierarchical way to
address the aforementioned shortcomings. This is achieved by advantageously
combining the functionality provided by the two community-driven tools
EasyBuild and Lmod.

The remainder of this paper is structured as follows: \ldots

\remark{issues with installing \& providing scientific software}
\remark{importance of appropriate tools \& community}


\section{Traditional approach}
\label{sec:traditional}
In this section, we describe common approaches that are traditionally used at
HPC sites to install scientific software and to deal with the growing number
of modules over the lifetime of an HPC system. We start our discussion with a
brief overview of the concept of module files and various module naming
schemes which are in use today, followed by a description of the underlying
module tools used by most sites. Next, we review commonly used workflows for
installing scientific software and finally summarize the shortcomings we have
identified with these approaches.

\subsection{Tools \& policies}

\subsubsection{Module files}
\label{sec:Module_files}

In essence, module files are a textual description of what needs to be
changed in the user's environment to make a particular software package
available. Such changes may include the adjustment of environment variables
such as \texttt{PATH} or setting additional package-specific variables, for
example, convenience variables pointing to the include and/or library paths.
In addition, module files typically include a brief one-line description of
the package displayed by \texttt{module whatis} as well as a longer help text
printed by \texttt{module help} to describe the basic usage, where to find
the package documentation, and whom to contact in case of usage problems
(this may be the site's application support team or the developers of the
package directly).

Module files are recursively searched for in directories specified by the
environment variable \texttt{MODULEPATH}. The name of a module is defined
as the full path to the corresponding module file, starting from the
specified search directory. For example, the module file located in
\begin{verbatim}
    <searchdir>/compiler/GCC/4.8.2
\end{verbatim}
provides a module for version 4.8.2 of the GNU Compiler Collection with the
name `\texttt{compiler/GCC/4.8.2}'.

The example above also shows one possible way to categorize modules by
placing their module files into appropriately named subdirectories (e.g.,
`\texttt{compiler}', `\texttt{mpi}', `\texttt{math}', etc.). Another common
option is to list the individual subdirectories in \texttt{MODULEPATH}, so
that the module names are shorter, but the modules are still nicely separated
in the output of module avail, for example:
\begin{verbatim}
    % module avail
    -------- <searchdir>/compiler --------
    GCC/4.8.2   Intel/14.0  Clang/3.4
    ---------- <searchdir>/mpi -----------
    OpenMPI/1.7.3 MVAPICH/1.9
\end{verbatim}

However, since packages such as MPI implementations are inherently tied to a
particular compiler and most often even a particular version, disambiguating
module names can be a daunting task. For example, for the three compilers
shown above, the corresponding Open\,MPI modules are often named as follows:
\begin{verbatim}
    OpenMPI/1.7.3-GCC-4.8.2
    OpenMPI/1.7.3-Intel-14.0
    OpenMPI/1.7.3-Clang-3.4
\end{verbatim}

Now considering a full-blown scientific software package like WRF compiled
with a particular compiler and linked against a particular MPI, module names
get even messier, for example:
\begin{verbatim}
    WRF/3.5-GCC-4.8.2-OpenMPI-1.7.3
\end{verbatim}
Note that such packages in many cases also depend on a set of mathematical
libraries, such as BLAS+ScaLAPACK+FFTW vs. ACML vs. Intel MKL, which in a
real scenario extends the module name even further.

A common solution to this issue is to define so-called \emph{toolchain}
modules, packaging a compiler, an MPI library, and one or more packages
providing linear algebra and FFT functionality. For example, a \texttt{goolf}
toolchain module may combine (i.e., implicitly load modules for) GCC,
Open\,MPI, OpenBLAS, ScaLAPACK and FFTW---each with a well-defined version. The
WRF module as shown above may the simply refer to a toolchain instead of the
individual packages:
\begin{verbatim}
    WRF/3.5-goolf-1.6.10
\end{verbatim}
The toolchain concept also offers the possibility to categorize modules by
toolchain, however, if a software package is availabel for multiple
toolchains, it will then show up in multiple sections of the \texttt{module
avail} output. The downside of using toolchains, however, is that users have
to be taught what is hidden behind the often rather cryptic toolchain names
and that a toolchain version has no direct relationship with the versions of
the encapsulated packages.

It should be pointed out that altough all of the approaches to name and
categorize module files presented above try to improve the overall
organization of the available modules, a typical module listing on an HPC
system can still be overwhelming, as the total number of modules can easily
be in the order of hundreds. Moreover, all traditional aproaches offer a
multitude of options for (especially novice) users to shoot themselves in the
foot, that is, to load modules which are incompatible to each other. While
module files in principle offer the possibility to specify conflicts and
thereby prevent loading of incompatible modules, all conflicting modules have
to be explicitly listed. For example, the Open\,MPI module built for GCC may
specify that it is incompatible with modules providing Intel or Clang
compilers. However, this means that these conflict specifications have to be
adjusted once an additional compiler (e.g., PGI) gets installed on the
system---which is clearly a maintenance nightmare from the perspective of a
system administrator.

\remark{briefly explain concept of module files, flat naming scheme +
variations (e.g. categorized, per toolchain, \ldots)}

\subsubsection{Modules tool}

\paragraph{Environment modules}

In the traditional environment modules
implementation~\cite{environment_modules_paper}, the \texttt{module} command
is implemented as a simple shell function (for Bourne-compatible shells) or
alias (for csh-compatible shells) which evaluates the commands printed by a
helper tool \texttt{modulecmd} to standard output. This helper tool
implements the actual functionality of identifying the requested command,
locating and parsing the corresponding module files, and generating the
commands necessary to modify the user's environment.

Over time, multiple implementations of the \texttt{modulecmd} helper tool
have been developed. The traditional version is written in C, using the Tcl
library to parse and evaluate module files. A second implementation, which
was never packaged as a release and is still marked as experimental by the
authors, is written in Tcl only. For the most part, these two implementations
offer identical functionality, however, they obviously suffer from the usual
problem of keeping the different code bases consistent, which sometimes leads
to subtle effects when switching between them. In addition, there exists a
fork of the Tcl-only implementation which has been heavily adjusted to meet
the requirements of the DEISA (Distributed European Infrastructure for
Supercomputing Applications) project.

\remark{briefly explain Tcl/C and Tcl-only environment module tools}

While all three implementations provide the desired basic functionality, they
are barely maintained and development progresses slowly. For example, there
has been no activity in the (publicly accessible) version control system of
the Tcl-only implementation for about two years. That is, new features such
as improved support for organizing modules in a hierarchical way are very
unlikely to happen any time soon.
\remark{works, but barely maintained, no further development/features, \ldots}

\markus{From what I can see, there are at least active discussions on the
modules-interest mailing list. However, the answers from the maintainers
suggest to me that they have no interest in changing things too much. What is
your impression? I'm also not sure whether we should really write this down
in a paper ;-)}

\markus{One possible reviewer comment may be: Why didn't you contribute a patch to
the existing project rather than reinventing the wheel? Do we have a good
answer to this? Robert, did you actually try to contribute and they rejected
to include the patch?}


\paragraph{Other}

???
\remark{cmod, \ldots}

\subsubsection{Installing scientific software}

???

\remark{manual, in-house scripting, 'Jim', little to no collaboration across
HPC sites (a few exceptions!)}

\markus{Does it make sense to also mention RPM and DEB packages? For a
regular software install, the downside is that only one version can be
available. But of course packaging systems can be combined with our proposed
approach to roll out the software on a bunch of nodes (see TACC).}

\subsection{Shortcomings \& unresolved issues}

???
\remark{manual creating of module files (consistency issues), repetitive \&
error-prone work of installing scientific software, \ldots}


\section{Hierarchical module naming scheme}
\label{sec:hierarchical}
%\remark{this section should outline the modern approach of using a hierarchical module naming scheme,
%highlighting how this makes things easier for end-users without limiting power users; forward references should
%be made to the EasyBuild \& Lmod sections when the consequences of using a module naming scheme like this are discussed}

The use of \emph{hierarchical naming schemes} is an excellent way to
help users avoid the pitfalls described earlier. This approach
makes it possible to organize environment modules in a more structured
way. The key idea is to make modules available in a step-by-step fashion as
their dependencies become part of the user environment.  Initially only a small
number of so-called \emph{core} modules are available to the user. Core modules are
those that do not depend on software choices available to the user; they are either
completely self-contained or depend only on basic system software. Examples include
modules for compilers, and statically linked software like debuggers.

These core modules extend the module search path
(\texttt{\small \$MODULEPATH}) when loaded to make additional modules visible, for
example the ones which are built with---and therefore depend on---the compiler
being loaded. Separate sub-directories for each version of each 
compiler store the software and module files that depend on
that compiler. For the
Open\,MPI example presented in Section~\ref{sec:Module_naming_scheme}, this means
that the user only sees the module for OpenMPI 1.7.3 that depends on
the current compiler of choice.

{\small
\begin{alltt}
    \textbf{% module avail}
    ------------ \emph{<prefix>}/Core ------------
    GCC/4.8.2   Intel/14.0  Clang/3.4
    \textbf{% module load GCC/4.8.2}
    \textbf{% module avail}
    ------------ \emph{<prefix>}/Core ------------
    GCC/4.8.2   Intel/14.0  Clang/3.4
    ----- <\emph{prefix}>/Compiler/GCC/4.8.2 -----
    OpenMPI/1.7.3\
\end{alltt}
}
\noindent
Such a hierarchy is not limited to a single level. The
module files for each MPI implementation, for example, can further extend
\texttt{\small \$MODULEPATH} to make visible the modules that depend on the currently loaded compiler and MPI stack.


\subsection{Advantages over traditional module naming schemes}
\label{sec:hierarchical_advantages}

%\markus{This sub-subsection IMHO only makes sense if there would be a second
%one. I propose to remove it and simply list the advantages as part of Section
%III.A.}

Using a hierarchical module naming scheme has a number of important
advantages. First, at any point in time, users see only the
modules which are meaningful in the current context. That is, the list of
available modules is much shorter and grouped by level of the hierarchy. This is easier for users to process when hundreds of modules are
provided (which is generally the case on large HPC systems).

Second, encoding the dependency chain in the module name is no longer
necessary. This leads to significantly shorter and more intuitive
module names, usually consisting of the software name and version, e.g.,
\texttt{\small WRF/3.5} instead of the long and cryptic module names shown in
Section~\ref{sec:Module_naming_scheme}.

Third, loading incompatible modules is much more difficult,
which eliminates a broad class of subtle errors that
are difficult to debug. This not only dramatically improves the user experience; it
also significantly reduces the time a user support team needs to spend on related
problems.

Finally, this structured organization of module files provides
a number of new opportunities to enhance the user experience. One significant
example is module \emph{swapping}: when the user chooses a new compiler or MPI
stack by executing a command like \texttt{\small module swap GCC Intel}, the module
system could (and should) automatically replace higher-level modules with new
versions compatible with the user's new lower-level selections.
\markus{I'm confused... Shouldn't it be ``replace higher-level modules ...
compatible with the user's new lower-level selections''?}\kenneth{depends on how you look
at the module tree, but I can see why your wording would be easier to follow}
We strongly believe
that this capability should be part of any robust module system; see
Section~\ref{sec:using_a_hierarchy}.

%\remark{basic users make less mistakes, expert users retain full freedom}

\subsection{Using a hierarchical module naming scheme}
\label{sec:using_a_hierarchy}

In theory, the commonly used Tcl/C and Tcl-only environment modules tools
both support the concept of hierarchical modules; this is because these tools
function in a way that is essentially independent of any particular choice in
organizing the module files. In practice, some difficulties when using these tools
with a module hierarchy come forward.

\subsubsection{Visibility of modules}
\label{sec:hierarchical_consequences_visibility}

In the hierarchical scheme we describe above, a module is not available (visible)
to the user until its lower-level \markus{Likewise ``lower-level''}\kenneth{ok} modules are loaded.  This
is by design, and has great advantages. But a module system should not leave to the
user the burden of locating needed modules within a complicated directory structure.  Instead, the module system should provide a built-in, natural mechanism for exposing hierarchies and
dependencies, and make it possible for users to load needed modules
without resorting to raw searches of the system's directory structure. In particular, the module system must provide a means for
displaying relevant information about modules that are outside the 
currently loaded \texttt{\small \$MODULEPATH}.

\subsubsection{Awareness of module path extensions}
\label{sec:hierarchical_consequences_extensions}

The modules tool should be aware of the changes  to the module path that
occur when loading modules in a hierarchical scheme. This is what makes
automatic module swapping possible. When the user executes \texttt{\small module swap} to replace one module with another, the tool needs to i) detect that a particular
module depends on the module being swapped out (this requires an awareness of
the hierarchical structure); ii) unload the dependent
modules and iii) afterwards
automatically (try to) replace them with equivalent compatible modules, taking into
account the correct order in which to (re)load those.

\subsubsection{Module availability on different paths in the hierarchy}
\label{sec:hierarchical_consequences_availability}

The modules tool also needs to take into account that it may not always be
possible to reload all dependent modules after swapping modules. When swapping
one compiler for another for example, it is possible that no compatible version of a
higher-level \markus{higher-level}\kenneth{ok} module is available. Unless the modules tool deals with this issue
appropriately, the user could end up stuck with a broken environment somewhere between
the original state (before the swap) and the intended end state (in which all
dependent modules are reloaded). Ideally, the modules tool should notify the user
when dependent modules cannot be reloaded. Moreover, if the modules tool can keep
track of which modules failed to reload it can also support the ability to revert
the swap, and restore the original set of loaded modules.

The existing Tcl/C and Tcl-only modules tools currently do not provide any of this
functionality. At one point, the Tcl-only tool did briefly support reloading
dependent modules after a \texttt{\small module swap}, but this capability later
disappeared because of incompatibility with the Tcl/C tool.

% moved to lmod.tex

%Finally, \remark{saved collections to set default compiler/MPI to avoid that
%novice users need to pick a compiler/MPI stack before they can get anything
%done}
%\kenneth{Robert: does this make sense to include it? or is this not
%related enough to a hierarchical scheme?}
%\robert{I would remove from Finally ... to here.  It is a mixing of
%  concepts. At TACC we provide a default set of modules which the user
%  can override with their own personal collection of modules}

%\markus{Somewhere, we should mention that the hierarchical module scheme as
%outlined above \textit{in principle} also works with C/Tcl or Tcl-only
%modules, but that specific features simplify usage a lot: ``stack-based'' swap,
%module deactivation, \texttt{module spider}, etc. But there is a small
%chicken/egg problem here \ldots}

\subsection{Maintaining a module hierarchy}
\label{sec:maintaining_a_hierarchy}

Next to the important issues related to using a module hierarchy discussed in the
previous section, an additional consideration is the potential impact on the user
support staff maintaining the modules.

In particular, a great deal of care must be exercised in constructing module files
for a hierarchical software stack; failing to do so can produce unexpected results
and inconsistent environments. One must think carefully about the necessary module
path extensions required at different levels of the hierarchy, take dependencies
into account, and consider the order in which users should load modules. Additionally, the
module files must be designed such that they use the short name visible to users
(e.g., \texttt{\small WRF/3.5}) rather than the `full' name relative to the top of
the hierarchy (e.g., \texttt{\small MPI/GCC/4.8.2/OpenMPI/1.7.3/WRF/3.5}).

Creating module files manually is already a tedious task; doing so by hand in a
hierarchical context only further complicates this. It should be clear that a modern 
HPC environment requires powerful, flexible, and reliable tools and techniques that
truly \emph{automate} the task of installing scientific software.

%% \subsection{Tools}
%%   \label{sec:tools_for_hierarchical}
In the sections that follow we describe how two community-driven tools work
together to achieve this task: \easybuild{}, a software build and installation
framework, and Lmod, a modern alternative to the Tcl-based environment modules
tools. Because our experience confirms the substantial benefits of a hierarchical
module naming scheme, and because we strongly believe that the HPC community is
ready to (and should) move in this direction, we focus primarily on the aspects
of these tools that support this approach.


\section{\easybuild{}: Automated software installation}
\label{sec:easybuild}

\subsection{Support for hierarchical module naming scheme}

\subsection{\easybuild{} as a platform for collaboration}


\section{Lmod: a modern modules tool}
\label{sec:lmod}

Lmod is a modern tool for consuming module files, with a strong focus on providing users
easy access to their (scientific) software stack, without hindering experts. It is the
handshake between the system administrators and end users, and delivers a powerful yet
flexible way of configuring and managing their working software stack. Lmod is feature-rich,
well supported, continously enhanced, and comes with a vibrant
community.   This talk will cover the main concepts in a module
system, the choice of module layout (hierarchical vs. flat),  The
new features of Lmod and and how to leverage the module system to
track how which software package user use and do not use.

\remark{actively maintained}

\subsection{Key features}

\begin{itemize}
    \item user-friendly
    \begin{itemize}
        \item ml
        \item sensible version ordering
        \item case-insensitive module avail
        \item list/avail (can) go(es) to stdout
        \item module load => swap if needed
        \item recursive unload (is default)
        \item spider cache => fast avail (\& spider)
    \end{itemize}
    \item hooks
    \item path priorities
    \item properties
    \item families
    \item load with version range
    \item pushenv
\end{itemize}
\remark{too long a list to include all? AP Robert: which ones do we really need to mention?}
\remark{module swap works}
\remark{module avail vs module spider}
\remark{(spider cache)}


%\section{Lessons learned}
%\label{sec:lessons}
%\remark{community: two is better than one, many outperform a few}

\kenneth{should this really be a separate section, or can we intertwine this somewhere? Robert?}


\section{Other aspects to EasyBuild \& Lmod}
\label{sec:communities_synergy}
\subsection{Communities}
\label{sec:communities}

Both \easybuild{} and Lmod are active developed, with development being
highly community-driven. Not only are new features implemented by the core
development teams following requests by their respective user communities, the
users themselves also significantly contribute by reporting bugs, sharing patches
for bugs they fixed themselves, and even by implementing new features
and subsequently sharing these enhancements so they can be included in later
releases.

Estimating the size of the \easybuild{} and Lmod communities is difficult.
Roughly a dozen different HPC sites actively contribute to the \easybuild{}, with
several others using it. The \easybuild{} mailing list has about 70 subscribers
to date, so a careful informed guess yields an estimate of over one hundred
users. The \easybuild{} IRC channel, one of the other main community hubs, has
about a dozen of daily active users. HPC sites all over the world are
actively using \easybuild{}, including various sites in Europe (e.g., J\"ulich
Supercomputing Centre (Germany)), Idaho National Laboratory (US), University of
Auckland (New Zealand), \ldots, as well as some commercial companies (e.g., Bayer).

The Lmod mailing list has about 50 people that are subscribed, but this
is known to represent no more than $20\%$ of the Lmod community at best. With a
couple of hundred HPC sites already providing Lmod to their users, the actual
numbers of Lmod users will be multiple thousands. Note that for example TACC is
providing Lmod as the only modules tool, already covering 15.000--20.000 users
over 3 different systems. Lmod has been downloaded several thousands of times
from the SourceForce code repository, across all versions, and is known to be
used by various HPC sites world-wide, including ones in North America (TACC,
Stanford, Harvard, Idaho National Labs) and Europe (Arctic
University of Norway, University of Warwick (UK)), and by commercial companies
like Total, among may others.

\subsection{Synergy}

Recently, a synergy has grown between \easybuild{} and Lmod. Both projects have a
community-oriented vision and highly value flexibility, which appeals to users
from both communities. Since \easybuild{} makes installing scientific software
significantly easier, user support teams can quickly generate lots of module
files, which can become overwhelming for users. This strengthens the need for a
modern modules tool that can efficiently deal with large collections of modules,
which led to the \easybuild{} community reaching out to the Lmod community.
Since then, support for using Lmod as a modules tool and support for hierarchical
module naming schemes has been integrated into \easybuild{}, and further
enhancements are planned (see also Section~\ref{sec:future_work}).

Likewise, Lmod has also been enhanced based on feedback by \easybuild{} users.
For example, the support for \texttt{pushenv} (see
Section~\ref{sec:lmod_pushenv}) was included based on discussions on the Lmod
mailing list by members of the \easybuild{} community. Recent versions of Lmod
have also included significant speed improvements which were triggered by issues
that were uncovered via \easybuild{}, e.g., the \texttt{module --terse avail}
command which provides machine-readable output and is used by \texttt{\small eb}, now
no longer reads the module files to obtain properties, since that information is not
required to compose the terse output text, resulting in significant speedups.

This synergy results in significantly better tools on both sides, further
pushing the potentially revolutionary nature of these tools.


\section{Future work}
\label{sec:future_work}
\remark{\itemize{
\item support for Lua module files in EasyBuild, along with family, properties, etc.
\item work out a deeper hierarchy that better matches the concept of compiler toolchains
\item support for EasyBuild on Cray and BlueGene systems
\item matrix support in Lmod
}}


\section{Related work}
\label{sec:related_work}
We are aware of a number of projects similar to \easybuild{}.

\emph{SWTools}~\cite{swtools, jones08} is an ``infrastructure for software
management" which was developed by National Center for Computational Sciences (NCCS)
and Oak Ridge National Lab (ORNL). It defines a structure for organizing a set of
(bash) scripts per supported software package, one per major installation step
(configuration, build, test, installation, etc.), allowing for easily reproducing 
software installations. It also takes care of generating module files, and is
able to update documentation listing the available software packages.
Only one version is publicly available, SWTools v1.0, which was released Jan.~1st 2011.

\emph{Smithy}~\cite{smithy} is a follow-up to SWTools, and is also being developed
at NCCS/ORNL. It is compatible with the SWTools infrastructure and hence can be
used as a drop-in replacement, but also supports an alternative approach using
\emph{formulas} following the well-established \emph{Homebrew}~\cite{homebrew} package
management system for Mac~OS~X. As such, it provides
supporting functionality readily available to be used in these formulas, and enables
code reuse across formulas. Smithy is publicly available alongside detailed
documentation. The development activity has slowed down significantly since
Sept.~2013, with occasional changes mostly focusing on bug fixes.
Smithy formulas are available for about 80 software
packages.

Another related project is the \emph{iVEC Build System (iBS)}, which is developed 
by iVEC in Australia. Similarly to \easybuild{}
it consists of a framework providing (some) commonly needed functionality, which
picks up so-called ``iBS files" that implement the install procedure for a 
particular software package. Both the main command (aptly named
\texttt{\small ibs}) and the iBS files are bash scripts, the latter being sourced by
the former as needed. At the time of writing iBS was not publicly available yet, but
the developers are known to be working towards a public release.

Finally, \emph{Spack (Supercomputing PACKage Manager)}~\cite{spack} is another
tool, written in Python, that is similar to \easybuild{} with respect to functionality and
goals. It provides a powerful and well-documented command line interface giving
control over which dependencies, software versions, compiler, architecture and
various options should be used for installing a particular software package.
Spack also supports some particularly useful options like automatically determining
whether an update for a particular software package is available
(by scraping the project's website), automatically completing
incomplete build specifications, optional/virtual dependencies, etc. Like
\easybuild{}, the Python codebase is object-oriented, enabling code reuse and
efficient maintainability; the \emph{packages} concept in it is quite similar to
easyblocks in \easybuild{}. At the time of writing, Spack included support for about
50 different software packages.

\markus{@Kenneth: One reviewer expressed severe concerns w.r.t. to the
following paragraph. Do you see a chance to mitigate (at least some) of the
statements?}\kenneth{fixed, reworded to 'configurable' instead of 'flexible',
which is more correct}
\easybuild{} differs from these tools in a number of ways. First and foremost, it is
more configurable than any of the other tools. Although they provide some support
for configuring their behavior, there is no control over certain aspects like, for example, the
module naming scheme being used. Second, several useful more advanced
features are missing in most of them, like automatic dependency resolution (Spack
being the exception here). Third, exactly reproducing previous installations is more
difficult since most of these tools do not employ separate specification files (again,
except for Spack); through easyconfig files this is particularly easy with \easybuild{} however.
Fourth, none of these tools has been able to get a sizable community going, which
has significant implications with respect to user contributions and the number of supported
software packages.
Other issues only apply to some of them, e.g., limited code
maintainability and reuse of code (SWTools and iBS), public availability of recent
versions (SWTools and iBS), and a lack of active development (SWTools and Smithy).

To the best of our knowledge, no recent module tools other than Lmod and the commonly
used Tcl-based tools discussed in Section~\ref{sec:env_modules_system} are
available. There are some alternative tools with comparable functionality, however.
Dotkit~\cite{dotkit} is a tool that supports loading and unloading package description 
files in a similar fashion to modules. Softenv~\cite{softenv} also provides similar
functionality, but uses a monolithic database to provide the package description data.
Both tools use a flat layout, however, making them incompatible with the concept of a
hierarchical naming scheme. Neither tool is widely used at HPC sites as
opposed to the environment modules system. On top of this, both projects are no longer
actively developed, with latest versions being made available in Aug.~2008 and
Mar.~2007, respectively.

Several HPC sites have been using a hierarchical module naming scheme for years,
including TACC~\cite{lmodSC11}, the Arctic University of Norway, the University
of Michigan, Calcul Qu\'ebec, and the University of Florida. Most of these also provide
Lmod to their users, but we are unaware of any sites using both \easybuild{} and Lmod
for efficiently handling software installations in a hierarchical context on
production systems. However, some
sites are actively looking into potentially applying such a methodology in the
foreseeable future, including HPC-UGent, the J\"ulich Supercomputing Centre, and
Stanford University.


\section{Conclusion}
\label{sec:conclusion}
\remark{\easybuild{} rules, Lmod is king!}
\markus{Future work: generate lua modulefiles from \easybuild{} including
family specifications to prevent loading two compiler modules at the same
time?}


\footnotesize
%\bibliographystyle{IEEEtran}
%\bibliography{ref}

\bibliographystyle{IEEEtranS}
% argument is your BibTeX string definitions and bibliography database(s)
\bibliography{IEEEabrv,main}

%%% Generated by IEEEtran.bst, version: 1.13 (2008/09/30)
%%\begin{thebibliography}{10}
%%\providecommand{\url}[1]{#1}
%%\csname url@samestyle\endcsname
%%\providecommand{\newblock}{\relax}
%%\providecommand{\bibinfo}[2]{#2}
%%\providecommand{\BIBentrySTDinterwordspacing}{\spaceskip=0pt\relax}
%%\providecommand{\BIBentryALTinterwordstretchfactor}{4}
%%\providecommand{\BIBentryALTinterwordspacing}{\spaceskip=\fontdimen2\font plus
%%\BIBentryALTinterwordstretchfactor\fontdimen3\font minus
%%  \fontdimen4\font\relax}
%%\providecommand{\BIBforeignlanguage}[2]{{%
%%\expandafter\ifx\csname l@#1\endcsname\relax
%%\typeout{** WARNING: IEEEtran.bst: No hyphenation pattern has been}%
%%\typeout{** loaded for the language `#1'. Using the pattern for}%
%%\typeout{** the default language instead.}%
%%\else
%%\language=\csname l@#1\endcsname
%%\fi
%%#2}}
%%\providecommand{\BIBdecl}{\relax}
%%\BIBdecl
%%
%%\bibitem{environment_modules_paper}
%%J.L.~Furlani, P.W.~Osel.
%%\newblock Abstract yourself with modules.
%%\newblock In \textit{LISA}, 1996, pp. 193--204.
%%
%%
%%%\end{thebibliography}

\end{document}
