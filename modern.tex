
\subsection{Hierarchical module naming scheme}

The idea behind a hierarchical module organization is to make different
variants of module files only visible on demand. That is, initially only a
small number of core modules are available to the user. Core modules are
modules which are independent of a particular compiler toolchain, that is,
either completely self-contained or only dependent on basic system software.
Examples of this category are compiler modules, modules providing statically
linked software (e.g., debuggers), and modules providing software packages
using only (certain) scripting languages.

By loading a compiler module, the \texttt{MODULEPATH} is then extended to
make all modules visible which are built with---and therefore depend on---the
corresponding compiler. That is, a separate module files directory is
maintained for each version of each compiler. For the Open\,MPI example
presented in Section~\ref{sec:Module_files}, this means that the user then
only sees a single module file for Open\,MPI 1.7.3 instead of three, the one
for the installation built with the corresponding compiler.

Obviously, such a hierarchy is not limited to a single level. That is, the
module files for different MPI implementations can further extend the
\texttt{MODULEPATH} to enable all modules which depend on both the loaded
compiler and the respective MPI installation.

\subsubsection{Advantages over traditional scheme}

A hierarchical module file organization has a number of significant
advantages: First, at any point in time, users are only presented with the
modules which are meaningful in the current context. That is, the list of
available modules is much shorter and therefore less overwhelming. Second,
encoding the dependency chain in the module name is no longer necessary for
modules enabling another hierarchy level, thereby leading to more intuitive
module names. And finally, loading of incompatible modules is automatically
avoided, preventing users from making simple mistakes which may lead to
subtle errors that are non-obvious and hard to debug.

\remark{basic users make less mistakes, expert users retain full freedom}



\subsection{\easybuild{}: building software with ease}

\subsubsection{\easybuild{} as a platform for collaboration}

\subsubsection{Support for hierarchical modules}



\subsection{Lmod: easy software access without hindering experts}

Lmod is a modern tool for consuming module files, with a strong focus on providing users
easy access to their (scientific) software stack, without hindering experts. It is the
handshake between the system administrators and end users, and delivers a powerful yet
flexible way of configuring and managing their working software stack. Lmod is feature-rich,
well supported, continously enhanced, and comes with a vibrant
community.   This talk will cover the main concepts in a module
system, the choice of module layout (hierarchical vs. flat),  The
new features of Lmod and and how to leverage the module system to
track how which software package user use and do not use.

\remark{actively maintained}

\subsubsection{Key features}

\begin{itemize}
    \item user-friendly
    \begin{itemize}
        \item ml
        \item sensible version ordering
        \item case-insensitive module avail
        \item list/avail (can) go(es) to stdout
        \item module load => swap if needed
        \item recursive unload (is default)
        \item spider cache => fast avail (\& spider)
    \end{itemize}
    \item hooks
    \item path priorities
    \item properties
    \item families
    \item load with version range
    \item pushenv
\end{itemize}
\remark{too long a list to include all? AP Robert: which ones do we really need to mention?}
\remark{module swap works}
\remark{module avail vs module spider}
\remark{(spider cache)}
