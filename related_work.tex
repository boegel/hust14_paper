We are aware of a couple of projects similar to \easybuild{}.

\emph{SWTools}~\cite{swtools, jones08} is an ``infrastructure for software
managment" which was developed at National Center for Computational Sciences (NCCS)
and Oak Ridge National Lab (ORNL). It defines a structure for organizing a set of
(bash) scripts per supported software package, one per major installation step
(configuration, build, test, installation, etc.), allowing for easily reproducing 
software installations. It also takes care of generating module files, and is
able to update documentation listing the available software packages.
Only one version is publicly available, SWTools v1.0, which was released Jan.~1st 2011.

\emph{Smithy}~\cite{smithy} is a follow-up to SWTools, and is also being developed
at NCCS/ORNL. It is compatible with the SWTools infrastructure and hence can be
used as a drop-in replacement, but also supports an alternative approach using
\emph{formulas} following the well-established \emph{Homebrew} package management 
system for Mac~OS~X\footnote{\url{http://brew.sh/}}. As such, it does provide
supporting functionality readily available to be used in these formulas, and enables
code reuse across formulas. Smithy is publicly available via
GitHub\footnote{https://github.com/AnthonyDiGirolamo/smithy} alongside detailed
documentation. The development activity has slowed down significantly since
Sept.~2013, with occasional changes mostly focusing on bug fixes.
Smithy formulas are available for about 80 software
packages\footnote{https://github.com/AnthonyDiGirolamo/smithy\_formulas}.

Another related project is the \emph{iVEC Build System (iBS)}, which is developed 
by iVEC\footnote{http://www.ivec.org/} in Australia. Similarly to \easybuild{}
it consists of a framework providing (some) commonly needed functionality, which
picks up so-called ``iBS files" that implement the install procedure for a 
particular software package as needed. Both the main command aptly named \texttt{ibs}
and the iBS files are bash scripts, the latter being sources by the former as needed.
At the time of writing iBS was not publicly available yet, but the developers are
known to be working towards a public release.

Finally, \emph{Spack (Supercomputing PACKage Manager)}~\cite{spack} is another
tool, written in Python, that is similar to \easybuild{} w.r.t.\ functionality and
goals. It provides a powerful and well-documented command line interface giving
control over which dependencies, software versions, compiler, architecture and
various options should be used for installing a particular software package.
Spack also supports some particularly useful options like automatically determining
whether an update for a particular software package is available
(by scraping the project's website), and automatically completing
incomplete build specifications, e.g., dependencies versions. Like
\easybuild{}, the Python codebase is object-oriented, enabling code reuse and
efficient maintainability; the \emph{packages} concept in is quite similar to
easyblocks in \easybuild{}. At the time of writing, Spack included support for about
30 different software packages.

\easybuild{} differs from these tools in a number of ways. First and foremost, it
provides more flexibility than any of these tools. Although they provide some support
for configuring their behavior, there is no control over certain aspects like the
module naming scheme being being used. Second, particularly useful more advanced
features like automatic dependency resolution is missing in each of them. Third, none
of these tools have gotten the same amount of traction as \easybuild{}, which has
significant implications w.r.t.\ user contributions and the number of supported
software packages. Fourth, because of a lack of build specification files (like
easyconfigs for \easybuild{}), exactly reproducing previous installations requires
re-using the exact command used before, thus requiring some kind of bookkeeping.
Other issues only apply to some of them, e.g., limited code
maintainability and reuse of code (SWTools and iBS), public availability of recent
versions (SWTools and iBS), a lack of active development (SWTools and Smithy) and
no support for generating module files (Spack).

To the best of our knowledge, no recent module tools other than Lmod and the commonly
used Tcl/C and Tcl-only tools discussed in Section~\ref{sec:env_modules_system} are
avaialble. There are some alternative tools with comparable functionality, however.
Dotkit~\cite{dotkit} ...
Softenv~\cite{softenv} ...
\kenneth{Robert: we need a couple of lines on dotkit and softenv here}
