\documentclass[10pt,xcolor={usenames,dvipsnames}]{beamer}

\usepackage{color}

\usetheme[headernav]{TACC}

\usecolortheme{TACCWhite}

\setbeamertemplate{headline}{}  %Remove the header
\setbeamertemplate{blocks}[rounded][shadow=true]


\begin{document}

%===============================================================================

\title{\textbf{Modern Scientific Software Management Using EasyBuild and Lmod}}
\author{%
    \textit{Markus Geimer (J{\"u}lich Supercomputing Centre, Germany)}\\~\\
    Kenneth Hoste (HPC-UGent, Ghent University, Belgium)\\~\\
    Robert McLay (Texas Advanced Computing Center, TX, USA)
}
\date{%
    \small%
    1\textsuperscript{st} International Workshop on HPC User Support Tools (HUST-14)\\%
    in conjunction with SC'14, New Orleans, LA, USA\\%
    November 21, 2014%
}

\begin{frame}[noframenumbering]
\titlepage
\end{frame}

%===============================================================================

% proper slide numbers in lower right-hand corner
\addtobeamertemplate{headline}{}{
    \usebeamerfont{headline}
    \usebeamercolor[fg]{headline}
    \hspace{63.5em}
    \vspace{-2.5em}
    \scriptsize{\insertframenumber/\inserttotalframenumber}
}

%===============================================================================

\begin{frame}{Motivation}
    HPC systems are typically used by large user communities.
    \begin{itemize}
        \item
            Widely varying demands
        \item
            Requires installation of many software packages
            \begin{itemize}
                \item
                    Sometimes identical/overlapping functionality (e.g., MPI libraries)
                \item
                    Multiple versions/builds
            \end{itemize}
    \end{itemize}
    \medskip
    This is challenging for both users and administrators.
    \begin{itemize}
        \item
            \emph{For users}: setting up the environment to use the desired software
            \begin{itemize}
                \item
                    Common solution: environment modules
            \end{itemize}
        \item
            \emph{For administrators}: installing software in a consistent way
            \begin{itemize}
                \item
                    Many HPC applications use non-standard build procedures
                \item
                    Successful installation steps often barely documented
                \item
                    Lack of sharing of good practices between HPC sites
            \end{itemize}
    \end{itemize}
\end{frame}

%===============================================================================

\begin{frame}{Environment modules}
\begin{itemize}
    \item
        Shell-independent way to modify a user's environment
    \item
        Provide `\texttt{module}' command, evaluating output of a helper tool
        \begin{itemize}
            \item
                Original implementations are Tcl-based (Tcl/C, Tcl-only)
            \item
                Lmod, implemented in Lua
        \end{itemize}
    \item
        Allows to, for example, list, load, unload, and swap modules
    \item
        Each module corresponds to a module file found in \texttt{\$MODULEPATH}
        \begin{itemize}
            \item
                Textual description of modifications to the user's environment\\
                (e.g., \texttt{\$PATH}, \texttt{\$CPATH},
                \texttt{\$LIBRARY\_PATH})
            \item
                Additional specifications such as conflicts, help texts, etc.
        \end{itemize}
\end{itemize}
\vspace*{-10pt}
\begin{center}
    \begin{minipage}{0.9\textwidth}
        \begin{exampleblock}{\footnotesize Example: available modules, loading a module}
            \scriptsize
            \ttfamily
            \$ module avail\\
            foo/1.0 \quad foo/1.7 \quad bar/4.2\\
            \$ module load foo\\
            \$ module list\\
            Currently Loaded Modulefiles:\\
            ~1) foo/1.7
        \end{exampleblock}
    \end{minipage}
\end{center}
\end{frame}

%===============================================================================

\begin{frame}{Flat module naming schemes (I)}
\begin{itemize}
    \item
        HPC systems often feature multiple compilers \& MPI libraries
        \begin{itemize}
            \item
                Packages built with different compilers/MPIs should not
                be mixed
        \end{itemize}
    \item
        Common solution: encode dependency in module name\\
        \enskip
        \begin{minipage}{0.9\textwidth}
            \begin{exampleblock}{\footnotesize Example: encoding compiler in name of MPI module}
                \scriptsize
                \ttfamily
                \$ module avail OpenMPI\\
                OpenMPI/1.7.3-GCC-4.8.2
                    \quad OpenMPI/1.7.3-Intel-14.0
            \end{exampleblock}
        \end{minipage}
    \smallskip
    \item
        Makes module names unwieldy for multiple dependencies\\
        \enskip
        \begin{minipage}{0.9\textwidth}
            \begin{exampleblock}{\footnotesize Example: long module names for application modules}
                \scriptsize
                \ttfamily
                \$ module avail WRF\\
                WRF/3.5-GCC-4.8.2-OpenMPI-1.7.3
                    \quad WRF/3.5-Intel-14.0-MVAPICH2-1.9
            \end{exampleblock}
        \end{minipage}
    \smallskip
    \item
        In many cases, packages additionally also depend on a set of
        mathematical libraries $\Rightarrow$ toolchains
        \begin{itemize}
            \item
                Cryptic toolchain names (e.g., `\texttt{goolf}')
            \item
                Toolchain version w/o direct relationship to encapsulated packages
        \end{itemize}
\end{itemize}
\end{frame}

%===============================================================================

\begin{frame}{Flat module naming schemes (II)}
\begin{itemize}
    \item
        Total number of modules easily $\cal{O}$(100)
    \item
        Categorization can improve clarity\\
        \enskip
        \begin{minipage}{0.9\textwidth}
            \begin{exampleblock}{\footnotesize Example: categorization via module search paths}
                \scriptsize
                \ttfamily
                \$ module avail\\
                ---------- <prefix>/compiler ----------\\
                GCC/4.8.2 \quad Intel/14.0 \quad Clang/3.4\\
                ------------- <prefix>/mpi ------------\\
                OpenMPI/1.7.3-GCC-4.8.2 \quad OpenMPI/1.7.3-Intel-14.0
            \end{exampleblock}
        \end{minipage}
    \smallskip
    \item
        Yet module listing can still be overwhelming
    \item
        Cumbersome to prevent loading incompatible modules
        \begin{itemize}
            \item
                Names of conflicting modules must be explicitly listed in module files
            \item
                Maintenance nightmare when adding/removing new packages
                conflicting with others
        \end{itemize}
\end{itemize}
\end{frame}

%===============================================================================

\begin{frame}{Installing scientific software}
\begin{itemize}
    \item
        Manual installation
        \begin{itemize}
            \item
                Heavily relies on manpower of (part of) user support staff
            \item
                Hard to enforce sharing of installation notes
        \end{itemize}
    \item
        Package managers (rpm, yum, apt-get, etc.)
        \begin{itemize}
            \item
                Limited support for installing multiple builds/versions of a
                package
            \item
                Not well suited to idiosyncrasies of scientific software
        \end{itemize}
    \item
        Scripting
        \begin{itemize}
            \item
                Loosely coupled collection of scripts to automate installations
            \item
                Often understood by only a few/single staff member(s)
            \item
                Even when publicly released, rarely flexible enough to
                accommodate other sites needs
        \end{itemize}
\end{itemize}
\begin{center}
    \begin{minipage}{0.9\textwidth}
        \begin{alertblock}{\small Wake-up call!}
            \footnotesize
            Although many HPC sites around the world face these problems,
            there is hardly any collaboration to address them!
        \end{alertblock}
    \end{minipage}
\end{center}
\end{frame}

%===============================================================================

\begin{frame}{Hierarchical module naming scheme}
    \textbf{Key idea}: make modules available step-by-step.
    \begin{itemize}
        \item
            Initially, only `core' modules (e.g., compilers) are visible
        \item
            These extend \texttt{\$MODULEPATH} on load, makes
            more modules available
    \end{itemize}
    \quad\quad
    \begin{minipage}{0.9\textwidth}
        \begin{exampleblock}{\footnotesize Example: a simple module hierarchy}
            \scriptsize
            \ttfamily
            \$ module avail\\
            ------------ <prefix>/Core ------------\\
            GCC/4.8.2 \quad Intel/14.0 \quad Clang/3.4\\
            \$ module load GCC/4.8.2\\
            \$ module avail\\
            ------------ <prefix>/Core ------------\\
            GCC/4.8.2 \quad Intel/14.0 \quad Clang/3.4\\
            ----- <prefix>/Compiler/GCC/4.8.2 -----\\
            OpenMPI/1.7.3
        \end{exampleblock}
    \end{minipage}

    \medskip
    Major advantages:
    \begin{itemize}
        \item
            Intuitive, short module names
        \item
            Only shows modules which are compatible in the current context
    \end{itemize}
\end{frame}

%===============================================================================

\begin{frame}{Challenges with using a module hierarchy}
\begin{itemize}
    \item
        Visibility of modules
        \begin{itemize}
            \item
                Initially only core modules show up in output of `\texttt{module avail}'
            \item
                How to locate packages w/o manually exploring the entire
                hierarchy?
        \end{itemize}
    \smallskip
    \item
        Awareness of changes to \texttt{\$MODULEPATH}
        \begin{itemize}
            \item
                Does swapping modules require reloading of other modules due
                to changes in the module search path?
        \end{itemize}
    \smallskip
    \item
        Module availability on different paths in the hierarchy
        \begin{itemize}
            \item
                What if a dependent module is not available in the target
                module search path after swapping a module lower in the
                hierarchy?
        \end{itemize}
    \smallskip
    \item
        Creating and maintaining a module hierarchy requires special care
        \begin{itemize}
            \item
                Appropriate hierarchy level, correctly handling dependencies, \ldots
        \end{itemize}
\end{itemize}
\end{frame}

%===============================================================================

\begin{frame}{Software build and installation tools}
\begin{itemize}
    \item
        SWTools (NICS/ORNL)
        \begin{itemize}
            \item
                Only one public release (2011)
        \end{itemize}
    \item
        Smithy (NICS/ORNL)
        \begin{itemize}
            \item
                Follow-up to SWTools, also supports formulas ($\sim$80 packages)
            \item
                Available on GitHub, since 2013 mostly bug fixes
        \end{itemize}
    \item
        iBS (iVEC)
        \begin{itemize}
            \item
                Not yet publicly available
        \end{itemize}
    \item
        Spack (LLNL)
        \begin{itemize}
            \item
                Powerful and well-documented command-line interface
            \item
                Supports $\sim$50 packages
            \item
                Available on GitHub
        \end{itemize}
\end{itemize}

\begin{center}
    \begin{minipage}{0.9\textwidth}
        \begin{alertblock}{\small Houston, we \emph{had} a problem!}
            \footnotesize
            \begin{itemize}
            \item No support for organizing modules hierarchically
            \item Lack of sizable communities up until now
            \end{itemize}
        \end{alertblock}
    \end{minipage}
\end{center}

\end{frame}

%===============================================================================

\begin{frame}{Module tools}
Different implementations of environment modules system:
\begin{itemize}
    \item
        Cmod (C based): now obsolete (last updated in 1998)
    \item
        Tcl-based implementations (Tcl/C or Tcl-only): common practice
    \item
        pymodule (Python based): experimental/incomplete
\end{itemize}
Tools similar to environment modules (no longer actively developed):
\begin{itemize}
    \item
        Dotkit: last updated in 2008
    \item
        Softenv: last updated in 2007
\end{itemize}

\begin{center}
    \begin{minipage}{0.9\textwidth}
        \begin{alertblock}{\small Houston, we \emph{had} a problem!}
            \footnotesize
            Module hierarchy usability challenges are not addressed by any of these.
        \end{alertblock}
    \end{minipage}
\end{center}
\end{frame}

%===============================================================================

\begin{frame}{Tools for dealing with a module hierarchy}
Adequate tools for dealing with a module hierarchy are indispensable.
\begin{itemize}
    \item
        Modules tool that users interact with must be \textit{hierarchy-aware}
    \item
        Complexity of maintaining a module hierarchy begs for \textit{automation}
\end{itemize}~\\

Two recent tools provide the necessary support:
\begin{itemize}    
    \item
        \textbf{Lmod}: \url{https://www.tacc.utexas.edu/tacc-projects/lmod}
        \begin{itemize}
            \item
                Specifically developed for using a hierarchical module
                tree while supporting flat module layouts
            \item
                Provides required hierarchy-specific features
        \end{itemize}
    \item
        \textbf{EasyBuild}: \url{http://hpcugent.github.io/easybuild}
        \begin{itemize}
            \item
                Automates software installation process, generates module files
            \item
                Provides full control over module naming scheme
            \item
                Includes support for organizing modules hierarchically
        \end{itemize}
\end{itemize}
\end{frame}

%===============================================================================

\begin{frame}{Lmod: a modern modules tool}
\begin{itemize}
    \item
        Drop-in alternative for Tcl-based module tools (a few edge cases)
    \item
        Improves user experience, without hindering experts
    \item
        Written in Lua, available since Oct'08, reads Tcl module files
    \item
        Frequent releases, driven by community demands and feedback
\end{itemize}
\quad\quad
\begin{minipage}{0.9\textwidth}
    \begin{exampleblock}{\footnotesize Example: swapping modules using \texttt{ml} shorthand in a module hierarchy}
        \scriptsize
        \ttfamily
        \$ ml\\
        Currently loaded modules:\\
        ~1) GCC/4.8.2 \quad 2) MPICH/3.1.1 \quad 3) FFTW/3.3.2\\
        \$ ml -GCC Clang\\
        The following have been reloaded:\\
        ~1) FFTW/3.3.2 \quad 2) MPICH/3.1.1\\
        \$ ml\\
        Currently loaded modules:\\
        ~1) Clang/3.4 \quad 2) MPICH/3.1.1 \quad 3) FFTW/3.3.2
    \end{exampleblock}
\end{minipage}
\end{frame}

%===============================================================================

\begin{frame}{Lmod: feature highlights}
\begin{itemize}
    \item
        \emph{Module hierarchy-aware} design and functionality
    \begin{itemize}
        \item
            Searching across entire module tree with `\texttt{spider}' subcommand
        \item
            Automatic reloading of dependent modules on `\texttt{module swap}'
        \item
            Marking missing dependent modules as inactive after `\texttt{module swap}'
    \end{itemize}
    \item
        Caching of module files, for responsive subcommands
        (e.g., \texttt{avail})
    \item
        Site-customizable behavior via provided hooks
    \item
        \texttt{ml} shorthand command, load/unload shortcuts
    \item
        Various other useful/advanced features, including:
    \begin{itemize}
        \item
            Case-insensitive `\texttt{avail}' subcommand
        \item
            Can send subcommand output to stdout (rather than to stderr)
        \item
            Defining module families (e.g., `\texttt{compiler}', `\texttt{mpi}')
        \item
            Assigning properties to modules (e.g., `\texttt{Phi-aware}')
        \item
            Stack-based definition of environment variables (\texttt{pushenv})
        \item
            User-definable collections of modules
    \end{itemize}
\end{itemize}
\end{frame}

%===============================================================================

\begin{frame}{EasyBuild: building software with ease}
\begin{figure}[centering]\includegraphics[height=1.2cm]{easybuild_logo.jpg}\end{figure}
\begin{itemize}
    \item
        Open source (GPLv2) framework for building and installing software
    \item
        Collection of Python packages and modules
    \item
        Original implementation by HPC-UGent, since 2009
    \item
        Thriving community: actively contributing, driving development
    \item
        New release every 4--6 weeks
        \begin{itemize}
            \item Latest release: EasyBuild v1.15.2 (Oct'14)
        \end{itemize}
    \item
        Supports over 500 different software packages
        \begin{itemize}
            \item
                Including CP2K, NAMD, NWChem, OpenFOAM, PETSc, QuantumESPRESSO,
                WRF, \ldots
        \end{itemize}
    \item
        Well documented: \url{http://easybuild.readthedocs.org}
\end{itemize}
\end{frame}


%===============================================================================

\begin{frame}{EasyBuild: feature highlights}
\begin{itemize}
    \item
        Fully autonomously building and installing (scientific) software
        \begin{itemize}
            \item
                Automatic dependency resolution
            \item
                Automatic generation of module files
        \end{itemize}
    \item
        Thorough logging of executed procedure
    \item
        Highly configurable, via config files/environment/command line
    \item
        Dynamically extendable with additional easyblocks, toolchains, etc.
    \item
        \emph{Support for module hierarchies}, via custom module naming scheme
\end{itemize}
\quad\quad
\begin{minipage}{0.9\textwidth}
    \begin{exampleblock}{\footnotesize Example: building/installing WRF \& dependencies \emph{with a single command} (!)}
        \scriptsize
        \ttfamily
        \$ eb WRF-3.5-goolf-1.4.10.eb --robot\\
        \$ module spider WRF\\
        ...\\
        This module can only be loaded through the following modules:\\
        GCC/4.7.2, OpenMPI/1.6.4
    \end{exampleblock}
\end{minipage}
\end{frame}

%===============================================================================

\begin{frame}{EasyBuild: high-level design overview}
\begin{itemize}
    \item
        EasyBuild \emph{framework}
        \begin{itemize}
            \item
                Core of EasyBuild
            \item
                Provides supporting functionality for building and installing software
        \end{itemize}
    \item
        \emph{easyblock}
        \begin{itemize}
            \item
                Python module
            \item
                Implements a (generic) software build/install procedure
        \end{itemize}
    \item
        \emph{easyconfig} file
        \begin{itemize}
            \item
                Build specification: software name/version, toolchain, etc.
        \end{itemize}
    \item
        Compiler \emph{toolchain}
        \begin{itemize}
            \item
                Compilers with accompanying libraries (MPI, BLAS/LAPACK, etc.)
        \end{itemize}
\end{itemize}

\medskip\quad\quad
\begin{minipage}{0.9\textwidth}
    \begin{block}{\small Putting it all together}
        \footnotesize
        The EasyBuild \emph{framework} leverages \emph{easyblocks} to
        automatically build and install (scientific) software using a
        particular \emph{compiler toolchain}, as specified by one or multiple
        \emph{easyconfig files}.
    \end{block}
\end{minipage}
\end{frame}

%===============================================================================

\begin{frame}{Example use case}

\quad\quad
\begin{minipage}{0.9\textwidth}
    \begin{exampleblock}{\footnotesize Build and install WRF in a module hierarchy}
        \scriptsize
        \ttfamily
        \$ export EASYBUILD\_MODULE\_NAMING\_SCHEME=MyHMNS\\
        \$ eb WRF-3.5-goolf-1.4.10.eb --robot\\
        \$ eb WRF-3.5-ictce-5.3.0.eb --robot
    \end{exampleblock}
\end{minipage}

\quad\quad
\begin{minipage}{0.9\textwidth}
    \begin{exampleblock}{\footnotesize List existing WRF modules, load GCC build}
        \scriptsize
        \ttfamily
        \$ module spider WRF\\
        ...\\
        \$ ml GCC OpenMPI WRF
    \end{exampleblock}
\end{minipage}

\quad\quad
\begin{minipage}{0.9\textwidth}
    \begin{exampleblock}{\footnotesize Swap to Intel build of WRF}
        \scriptsize
        \ttfamily
        \$ ml -GCC -OpenMPI +icc +ifort +impi\\
        The following have been reloaded:\\
        ...
    \end{exampleblock}
\end{minipage}

\end{frame}

%===============================================================================

\begin{frame}{EasyBuild and Lmod communities}
\begin{itemize}
    \item
        Both EasyBuild and Lmod have vibrant communities
    \item
        Estimating their sizes is difficult, though
        \begin{itemize}
            \item
                EasyBuild
                \begin{itemize}
                    \item
                        Over 80 subscribers to mailing list
                    \item
                        About a dozen active members on \texttt{\#easybuild} IRC channel
                    \item
                        Users and contributors at HPC sites around the world\\
                        (e.g., JSC, Stanford, U. Auckland, Bayer AG, \ldots)
                    \item
                        Six 3-day EasyBuild `hackathons', at various European HPC sites
                \end{itemize}
            \item
                Lmod
                \begin{itemize}
                    \item
                        $\sim$50 subscribers to mailing list
                    \item
                        Deployed by a couple of hundred HPC sites\\
                        (e.g., Stanford, Harvard, TACC, U. Warwick,
                        JSC, Total, NASA, \ldots)
                    \item
                        Number of users $\cal{O}$(10,000)
                \end{itemize}
        \end{itemize}
    \item
        Many sites/users contribute by
        \begin{itemize}
            \item
                Requests, suggestions, and bug reports
            \item
                Sharing patches and implementing new features
        \end{itemize}
\end{itemize}
\end{frame}

%===============================================================================

\begin{frame}{Synergy between EasyBuild and Lmod}
\begin{itemize}
    \item
        EasyBuild can easily build and install hundreds of packages
        \begin{itemize}
            \item[$\Rightarrow$]
                Lots of modules, overwhelming for users
        \end{itemize}
    \item
        Lmod's support for hierarchical modules trees can help
        \begin{itemize}
            \item[$\Rightarrow$]
                Support for using Lmod and hierarchical module naming schemes
                was added to EasyBuild
        \end{itemize}
    \item
        EasyBuild uncovered performance issues in Lmod
        \begin{itemize}
            \item[$\Rightarrow$]
                Lmod has significantly improved the speed of certain operations \\
                (e.g., \texttt{module --terse avail})
        \end{itemize}
    \item
        Feature requests from the EasyBuild community
        \begin{itemize}
            \item[$\Rightarrow$]
                Lmod has added new functionality, for example stack-based definition of
                environment variables using `\texttt{pushenv}'
        \end{itemize}
\end{itemize}

\medskip\quad\quad
\begin{minipage}{0.9\textwidth}
    \begin{block}{\small Bottom line}
        \footnotesize
        Synergy between EasyBuild and Lmod has made both tools significantly better!
    \end{block}
\end{minipage}
\end{frame}

%===============================================================================

\begin{frame}{Future work}
\begin{itemize}
    \item
        Add support for Lmod-specific features in EasyBuild
        \begin{itemize}
            \item
                Properties
            \item
                Families
            \item
                Non-strict version loads: \texttt{load(atleast("GCC","4.8"))}
        \end{itemize}
    \smallskip
    \item
        The hierarchy concept needs further investigation
        \begin{itemize}
            \item
                Lmod currently supports a single hierarchy:
                Core $\Rightarrow$ Compiler $\Rightarrow$ MPI
            \item
                Traditional EasyBuild toolchains also contain multiple math
                libraries\\
                $\Rightarrow$ Multi-dimensional module hierarchy (``matrix'')
        \end{itemize}
    \smallskip
    \item
        Improve EasyBuild's dependency resolution mechanism to support
        subtoolchains
    \smallskip
    \item
        Extend EasyBuild to support multiple module naming schemes concurrently
        (e.g., to gradually move from one layout to another)
\end{itemize}
\end{frame}

%===============================================================================

\begin{frame}{Conclusion}
\begin{itemize}
    \item
        Building software for HPC systems is difficult and painful
    \item
        Flat module layout is common practice, but has many drawbacks
        \begin{itemize}
            \item
                Overwhelming number of modules
            \item
                Unwieldy module names due to dependencies
            \item
                Loading compatible modules is the user's responsibility
        \end{itemize}
    \item
        Hierarchical module organization helps significantly
        \begin{itemize}
            \item
                Fewer visible modules (only compatible ones), short module names
            \item
                Requires adequate tool support, however
        \end{itemize}
    \item
        EasyBuild
        \begin{itemize}
            \item
                Automates building and installing (scientific) software, incl.~modules
            \item
                Supports maintaining hierarchical module trees
            \item
                Helps to share knowledge between HPC centers
        \end{itemize}
    \item
        Lmod
        \begin{itemize}
            \item
                Provides required hierarchy-specific features (and a lot more)
        \end{itemize}
    \item
        Both tools significantly benefit from collaboration and
        communities
\end{itemize}
\end{frame}

%===============================================================================

\begin{frame}{Thank you!}
\begin{center}
    \Huge
    Questions?
\end{center}
\end{frame}

\end{document}
