\documentclass[10pt,xcolor={usenames,dvipsnames}]{beamer}

\usepackage{color}

\usetheme[headernav]{TACC}

\usecolortheme{TACCWhite}

\setbeamertemplate{headline}{}  %Remove the header
\setbeamertemplate{blocks}[rounded][shadow=true]

\setbeamerfont{block title}{size=\small}
\setbeamerfont{block body}{size=\footnotesize}
\setbeamerfont{block title example}{size=\footnotesize}
\setbeamerfont{block body alerted}{size=\scriptsize} % 'body example' doesn't work...


\begin{document}

%===============================================================================

\title{Modern Scientific Software Management Using EasyBuild and Lmod}
\author{%
    Markus Geimer\inst{1}%
    \and%
    Kenneth Hoste\inst{2}%
    \and
    Robert McLay\inst{3}%
}
\institute{%
    \inst{1} J{\"u}lich Supercomputing Centre, Germany%
    \and%
    \inst{2} Ghent University, Belgium%
    \and%
    \inst{3} Texas Advanced Computing Center, TX, USA%
}
\date{%
    \small%
    1\textsuperscript{st} International Workshop on HPC User Support Tools\\%
    in conjunction with SC'14, New Orleans, LA, USA\\%
    November 21, 2014%
}

\frame{\titlepage}

%===============================================================================

\begin{frame}{Motivation}
\begin{itemize}
    \item
        HPC systems are typically used by large user communities
        \begin{itemize}
            \item
                Widely varying demands
            \item
                Requires installation of many software packages
                \begin{itemize}
                    \item
                        Sometimes identical/overlapping functionality
                    \item
                        Multiple versions
                \end{itemize}
        \end{itemize}
    \vspace{1ex}
    \item
        Challenging for both users and administrators
        \begin{itemize}
            \item
                Users: Setting up the environment to use the desired software
                \begin{itemize}
                    \item
                        Common solution: Environment modules
                \end{itemize}
            \item
                Administrators: Consistently install software
                \begin{itemize}
                    \item
                        Many HPC codes use non-standard build procedures
                    \item
                        Successful installation steps often rarely documented
                    \item
                        \ldots not to mention shared between HPC sites
                \end{itemize}
        \end{itemize}
\end{itemize}
\end{frame}

%===============================================================================

\begin{frame}{Environment Modules}
\begin{itemize}
    \item
        Shell-independent way to modify a user's environment
    \item
        Provide `\texttt{module}' command
        \begin{itemize}
            \item
                Implemented as shell function (sh) or alias (csh)
            \item
                Evaluates commands printed to stdout by a helper tool
                \begin{itemize}
                    \item
                        Original implementation written in C/Tcl
                    \item
                        Alternative Tcl-only variants
                    \item
                        Lmod, implemented in Lua
                \end{itemize}
        \end{itemize}
    \item
        Allows to, e.g., list, load, unload, and swap modules
    \item
        Each module corresponds to a modulefile found in \texttt{\$MODULEPATH}
        \begin{itemize}
            \item
                Textual description of modifications to the user's environment\\
                (e.g., \texttt{\$PATH}, \texttt{\$CPATH},
                \texttt{\$LIBRARY\_PATH})
            \item
                Additional specifications such as conflicts, help texts, etc.
        \end{itemize}
\end{itemize}
\vspace*{-5pt}
\begin{center}
    \begin{minipage}{0.9\textwidth}
        \begin{exampleblock}{Example}
            \ttfamily
            \% module avail\\
            foo/1.0 \enskip foo/1.7 \enskip bar/4.2\\
            \% module load foo\\
            \% module list\\
            Currently Loaded Modulefiles:\\
            ~1) foo/1.7
        \end{exampleblock}
    \end{minipage}
\end{center}
\end{frame}

%===============================================================================

\begin{frame}{Flat Module Naming Schemes (I)}
\begin{itemize}
    \item
        HPC systems often feature multiple compilers \& MPI libraries
        \begin{itemize}
            \item
                Packages built with different compilers/MPIs cannot
                be mixed
        \end{itemize}
    \item
        Common solution: encode dependency in module name\\
        \enskip
        \begin{minipage}{0.9\textwidth}
            \begin{exampleblock}{Example}
                \ttfamily
                \% module avail OpenMPI\\
                OpenMPI/1.7.3-GCC-4.8.2
                    \quad OpenMPI/1.7.3-Intel-14.0
            \end{exampleblock}
        \end{minipage}
    \smallskip
    \item
        Makes module names unwieldy for multiple dependencies\\
        \enskip
        \begin{minipage}{0.9\textwidth}
            \begin{exampleblock}{Example}
                \ttfamily
                \% module avail WRF\\
                WRF/3.5-GCC-4.8.2-OpenMPI-1.7.3
                    \quad WRF/3.5-Intel-14.0-MVAPICH2-1.9
            \end{exampleblock}
        \end{minipage}
    \smallskip
    \item
        In many cases, packages additionally also depend on a set of
        mathematical libraries $\Rightarrow$ toolchains
        \begin{itemize}
            \item
                Cryptic toolchain names
                \begin{itemize}
                    \item
                        E.g., \texttt{goolf} = GCC+Open\,MPI+OpenBLAS+ScaLAPACK+FFTW
                \end{itemize}
            \item
                Toolchain versions w/o direct relationship to encapsulated packages
        \end{itemize}
\end{itemize}
\end{frame}

%===============================================================================

\begin{frame}{Flat Module Naming Schemes (II)}
\begin{itemize}
    \item
        Total number of modules easily \cal{O}(100)
    \item
        Categorization can improve clarity\\
        \enskip
        \begin{minipage}{0.9\textwidth}
            \begin{exampleblock}{Example}
                \ttfamily
                \% module avail\\
                ----- <prefix>/compiler -----\\
                GCC/4.8.2 \quad Intel/14.0 \quad Clang/3.4\\
                -------- <prefix>/mpi --------\\
                OpenMPI/1.7.3-GCC-4.8.2 \quad OpenMPI/1.7.3-Intel-14.0
            \end{exampleblock}
        \end{minipage}
    \smallskip
    \item
        But module listing can still be overwhelming
    \item
        Cumbersome to prevent loading incompatible modules
        \begin{itemize}
            \item
                Conflicting modules have to be explicitly listed in modulefiles
            \item
                Maintenance nightmare when adding/removing new conflicting
                packages
        \end{itemize}
\end{itemize}
\end{frame}

%===============================================================================

\begin{frame}{Installing Scientific Software}
\begin{itemize}
    \item
        Manual installation
        \begin{itemize}
            \item
                Relies on (part of) user support staff
            \item
                Hard to enforce sharing of installation notes
        \end{itemize}
    \item
        Package managers (rpm, yum, apt-get, etc.)
        \begin{itemize}
            \item
                Limited support for installing multiple builds/versions of a
                package
            \item
                Little support for common procedures $\Rightarrow$ copy/paste
        \end{itemize}
    \item
        Scripting
        \begin{itemize}
            \item
                Losely coulped collection of scripts to automate installations
            \item
                Often understood by only a few/single staff member(s)
            \item
                Even when publicly released, rarely flexible enough to
                accommodate other sites needs
        \end{itemize}
\end{itemize}
\begin{center}
    \begin{minipage}{0.9\textwidth}
        \begin{alertblock}{Wake-up call!}
            \footnotesize
            Although many HPC sites around the world face these problems,
            there is hardly any collaboration to address them!
        \end{alertblock}
    \end{minipage}
\end{center}
\end{frame}

%===============================================================================

\begin{frame}{Hierarchical Module Naming Scheme}
\begin{itemize}
    \item
        Key idea: Make modules available step-by-step
        \begin{itemize}
            \item
                Initially, only core modules (e.g., compilers) are visible
            \item
                When loaded, these modules extend \texttt{\$MODULEPATH} to make
                dependent modules available
        \end{itemize}
        \enskip
        \begin{minipage}{0.9\textwidth}
            \begin{exampleblock}{Example}
                \ttfamily
                \% module avail\\
                ------------ <prefix>/Core ------------\\
                GCC/4.8.2 \quad Intel/14.0 \quad Clang/3.4\\
                \% module load GCC/4.8.2\\
                \% module avail\\
                ------------ <prefix>/Core ------------\\
                GCC/4.8.2 \quad Intel/14.0 \quad Clang/3.4\\
                ----- <prefix>/Compiler/GCC/4.8.2 -----\\
                OpenMPI/1.7.3
            \end{exampleblock}
        \end{minipage}
    \smallskip
    \item
        Major advantages:
        \begin{itemize}
            \item
                Intuitive, short module names
            \item
                Only shows modules which are meaningful in the current context\\
                $\Rightarrow$ Prevents many user mistakes
        \end{itemize}
\end{itemize}
\end{frame}

%===============================================================================

\begin{frame}{Challenges When Deploying a HMNS}
\begin{itemize}
    \item
        Visibility of modules
        \begin{itemize}
            \item
                How to locate packages w/o manually exploring the entire
                hierarchy?
        \end{itemize}
    \item
        Awareness of \texttt{\$MODULEPATH} extensions
        \begin{itemize}
            \item
                Does swapping a module require reloading of other modules due
                to changes in the module search path?
        \end{itemize}
    \item
        Module availability on different paths in the hierarchy
        \begin{itemize}
            \item
                What if a dependent module is not available in the target
                module search path after swapping a module lower in the
                hierarchy?
        \end{itemize}
    \item
        Creating and maintaining a module hierarchy requires even more care
\end{itemize}
\begin{center}
    \begin{minipage}{0.9\textwidth}
        \begin{alertblock}{Houston, we've had a problem!}
            \footnotesize
            \begin{itemize}
                \item
                    None of the usability challenges is addressed by the
                    C/Tcl or Tcl-only modules implementation
                \item
                    None of existing scientific software installation tools
                    supports hierarchical modules
            \end{itemize}
        \end{alertblock}
    \end{minipage}
\end{center}
\end{frame}

%===============================================================================

\end{document}
