\documentclass[10pt,xcolor={usenames,dvipsnames}]{beamer}

\usepackage{color}

\usetheme[headernav]{TACC}

\usecolortheme{TACCWhite}

\setbeamertemplate{headline}{}  %Remove the header
\setbeamertemplate{blocks}[rounded][shadow=true]

\setbeamerfont{block title}{size=\small}
\setbeamerfont{block body}{size=\footnotesize}
\setbeamerfont{block title example}{size=\footnotesize}
\setbeamerfont{block body alerted}{size=\scriptsize} % 'body example' doesn't work...


\begin{document}

%===============================================================================

\title{\textbf{Modern Scientific Software Management Using EasyBuild and Lmod}}
\author{%
    Markus Geimer\inst{1}%
    \and%
    Kenneth Hoste\inst{2}%
    \and
    Robert McLay\inst{3}%
}
\institute{%
    \inst{1} J{\"u}lich Supercomputing Centre, Germany%
    \and%
    \inst{2} HPC-UGent, Ghent University, Belgium%
    \and%
    \inst{3} Texas Advanced Computing Center, TX, USA%
}
\date{%
    \small%
    1\textsuperscript{st} International Workshop on HPC User Support Tools (HUST-14)\\%
    in conjunction with SC'14, New Orleans, LA, USA\\%
    November 21, 2014%
}

\begin{frame}[noframenumbering]
\titlepage
\end{frame}

%===============================================================================

% proper slide numbers in lower right-hand corner
\addtobeamertemplate{headline}{}{
    \usebeamerfont{headline}
    \usebeamercolor[fg]{headline}
    \hspace{63.5em}
    \vspace{-2.5em}
    \scriptsize{\insertframenumber/\inserttotalframenumber}
}

%===============================================================================

\begin{frame}{Motivation}
    HPC systems are typically used by large user communities.
    \begin{itemize}
        \item
            Widely varying demands
        \item
            Requires installation of many software packages
            \begin{itemize}
                \item
                    Sometimes identical/overlapping functionality (e.g., MPI libraries)
                \item
                    Multiple versions/builds
            \end{itemize}
    \end{itemize}
    \medskip
    This is challenging for both users and administrators.
    \begin{itemize}
        \item
            \emph{For users}: setting up the environment to use the desired software
            \begin{itemize}
                \item
                    Common solution: environment modules
            \end{itemize}
        \item
            \emph{For administrators}: installing software in a consistent way
            \begin{itemize}
                \item
                    Many HPC applications use non-standard build procedures
                \item
                    Successful installation steps often barely documented
                \item
                    \ldots not to mention sharing of good practices between HPC sites
            \end{itemize}
    \end{itemize}
\end{frame}

%===============================================================================

\begin{frame}{Environment modules}
\begin{itemize}
    \item
        Shell-independent way to modify a user's environment
    \item
        Provide `\texttt{module}' command, evaluating output of a helper tool
        \begin{itemize}
            \item
                Original implementation written in C/Tcl
            \item
                Alternative Tcl-only variants
            \item
                Lmod, implemented in Lua
        \end{itemize}
    \item
        Allows to, e.g., list, load, unload, and swap modules
    \item
        Each module corresponds to a modulefile found in \texttt{\$MODULEPATH}
        \begin{itemize}
            \item
                Textual description of modifications to the user's environment\\
                (e.g., \texttt{\$PATH}, \texttt{\$CPATH},
                \texttt{\$LIBRARY\_PATH})
            \item
                Additional specifications such as conflicts, help texts, etc.
        \end{itemize}
\end{itemize}
\vspace*{-10pt}
\begin{center}
    \begin{minipage}{0.9\textwidth}
        \begin{exampleblock}{Example: available modules, loading a module}
            \ttfamily
            \$ module avail\\
            foo/1.0 \quad foo/1.7 \quad bar/4.2\\
            \$ module load foo\\
            \$ module list\\
            Currently Loaded Modulefiles:\\
            ~1) foo/1.7
        \end{exampleblock}
    \end{minipage}
\end{center}
\end{frame}

%===============================================================================

\begin{frame}{Flat module naming schemes (I)}
\begin{itemize}
    \item
        HPC systems often feature multiple compilers \& MPI libraries
        \begin{itemize}
            \item
                Packages built with different compilers/MPIs can not
                be mixed
        \end{itemize}
    \item
        Common solution: encode dependency in module name\\
        \enskip
        \begin{minipage}{0.9\textwidth}
            \begin{exampleblock}{Example: encoding compiler in name of MPI module}
                \ttfamily
                \$ module avail OpenMPI\\
                OpenMPI/1.7.3-GCC-4.8.2
                    \quad OpenMPI/1.7.3-Intel-14.0
            \end{exampleblock}
        \end{minipage}
    \smallskip
    \item
        Makes module names unwieldy for multiple dependencies\\
        \enskip
        \begin{minipage}{0.9\textwidth}
            \begin{exampleblock}{Example: long module names for application modules}
                \ttfamily
                \$ module avail WRF\\
                WRF/3.5-GCC-4.8.2-OpenMPI-1.7.3
                    \quad WRF/3.5-Intel-14.0-MVAPICH2-1.9
            \end{exampleblock}
        \end{minipage}
    \smallskip
    \item
        In many cases, packages additionally also depend on a set of
        mathematical libraries $\Rightarrow$ toolchains
        \begin{itemize}
            \item
                Cryptic toolchain names (e.g., `\texttt{goolfc}')
            \item
                Toolchain versions w/o direct relationship to encapsulated packages
        \end{itemize}
\end{itemize}
\end{frame}

%===============================================================================

\begin{frame}{Flat module naming schemes (II)}
\begin{itemize}
    \item
        Total number of modules easily \cal{O}(100)
    \item
        Categorization can improve clarity\\
        \enskip
        \begin{minipage}{0.9\textwidth}
            \begin{exampleblock}{Example: categorization via module search paths}
                \ttfamily
                \$ module avail\\
                ---------- <prefix>/compiler ----------\\
                GCC/4.8.2 \quad Intel/14.0 \quad Clang/3.4\\
                ------------- <prefix>/mpi ------------\\
                OpenMPI/1.7.3-GCC-4.8.2 \quad OpenMPI/1.7.3-Intel-14.0
            \end{exampleblock}
        \end{minipage}
    \smallskip
    \item
        Yet module listing can still be overwhelming
    \item
        Cumbersome to prevent loading incompatible modules
        \begin{itemize}
            \item
                Conflicting modules have to be explicitly listed in modulefiles
            \item
                Maintenance nightmare when adding/removing new packages
                conflicting with others
        \end{itemize}
\end{itemize}
\end{frame}

%===============================================================================

\begin{frame}{Installing scientific software}
\begin{itemize}
    \item
        Manual installation
        \begin{itemize}
            \item
                Relies on (part of) user support staff
            \item
                Hard to enforce sharing of installation notes
        \end{itemize}
    \item
        Package managers (rpm, yum, apt-get, etc.)
        \begin{itemize}
            \item
                Limited support for installing multiple builds/versions of a
                package
            \item
                Little support for common procedures $\Rightarrow$ copy/paste
        \end{itemize}
    \item
        Scripting
        \begin{itemize}
            \item
                Losely coulped collection of scripts to automate installations
            \item
                Often understood by only a few/single staff member(s)
            \item
                Even when publicly released, rarely flexible enough to
                accommodate other sites needs
        \end{itemize}
\end{itemize}
\begin{center}
    \begin{minipage}{0.9\textwidth}
        \begin{alertblock}{Wake-up call!}
            \footnotesize
            Although many HPC sites around the world face these problems,
            there is hardly any collaboration to address them!
        \end{alertblock}
    \end{minipage}
\end{center}
\end{frame}

%===============================================================================

\begin{frame}{Hierarchical module naming scheme}
    Key idea: make modules available step-by-step.
    \begin{itemize}
        \item
            Initially, only `core' modules (e.g., compilers) are visible
        \item
            These extend \texttt{\$MODULEPATH} on load, to make
            more modules available
    \end{itemize}
    \quad\quad
    \begin{minipage}{0.9\textwidth}
        \begin{exampleblock}{Example: a simple module hierarchy}
            \ttfamily
            \$ module avail\\
            ------------ <prefix>/Core ------------\\
            GCC/4.8.2 \quad Intel/14.0 \quad Clang/3.4\\
            \$ module load GCC/4.8.2\\
            \$ module avail\\
            ------------ <prefix>/Core ------------\\
            GCC/4.8.2 \quad Intel/14.0 \quad Clang/3.4\\
            ----- <prefix>/Compiler/GCC/4.8.2 -----\\
            OpenMPI/1.7.3
        \end{exampleblock}
    \end{minipage}

    \medskip
    Major advantages:
    \begin{itemize}
        \item
            Intuitive, short module names
        \item
            Only shows modules which are meaningful in the current context
    \end{itemize}
\end{frame}

%===============================================================================

\begin{frame}{Challenges with using a module hierarchy}
\begin{itemize}
    \item
        Visibility of modules
        \begin{itemize}
            \item
                How to locate packages w/o manually exploring the entire
                hierarchy?
        \end{itemize}
    \item
        Awareness of \texttt{\$MODULEPATH} extensions
        \begin{itemize}
            \item
                Does swapping a module require reloading of other modules due
                to changes in the module search path?
        \end{itemize}
    \item
        Module availability on different paths in the hierarchy
        \begin{itemize}
            \item
                What if a dependent module is not available in the target
                module search path after swapping a module lower in the
                hierarchy?
        \end{itemize}
    \item
        Creating and maintaining a module hierarchy requires even more care
\end{itemize}
\begin{center}
    \begin{minipage}{0.9\textwidth}
        \begin{alertblock}{Houston, we've \emph{had} a problem!}
            \footnotesize
            \begin{itemize}
                \item
                    None of the usability challenges is addressed by the
                    C/Tcl or Tcl-only modules implementation
                \item
                    None of existing scientific software installation tools
                    supports hierarchical modules
            \end{itemize}
        \end{alertblock}
    \end{minipage}
\end{center}
\end{frame}

%===============================================================================

\begin{frame}{Tools for using a module hierarchy}
    \begin{itemize}
        \item adequate tools for dealing with a module hierarchy are \emph{indispensable}
        \item EasyBuild
        \item Lmod
    \end{itemize}
\end{frame}

%===============================================================================

\begin{frame}{EasyBuild: building software with ease}
\begin{figure}[centering]
    \includegraphics[height=1.2cm]{easybuild_logo.jpg}
\end{figure}
\begin{itemize}
    \item
        Collection of Python packages and modules
        \begin{itemize}
            \item
                Licensed under GPLv2
        \end{itemize}
    \item
        Original implementation by HPC-UGent, since 2009
    \item
        Now a thriving community
        \begin{itemize}
            \item Driving development
            \item Actively contributing
        \end{itemize}
    \item
        New release every 4--6 weeks
        \begin{itemize}
            \item Latest release: EasyBuild v1.15.2 (Oct'14)
        \end{itemize}
    \item
        Well documented: \url{http://easybuild.readthedocs.org}
\end{itemize}
\end{frame}

%===============================================================================

\begin{frame}{EasyBuild: high-level design overview}
\begin{itemize}
    \item
        EasyBuild \emph{framework}
        \begin{itemize}
            \item
                Core of EasyBuild
            \item
                Provides supporting functionality
        \end{itemize}
    \item
        EasyBuild \emph{easyblock}
        \begin{itemize}
            \item
                Python module
            \item
                Implements a (generic) build procedure
        \end{itemize}
    \item
        EasyBuild \emph{easyconfig}
        \begin{itemize}
            \item
                Build specification: software name/version, toolchain, etc.
        \end{itemize}
\end{itemize}

\medskip\quad\quad
\begin{minipage}{0.9\textwidth}
    \begin{block}{Putting it all together}
        The EasyBuild \emph{framework} leverages \emph{easyblocks} to
        \textbf{automatically build and install (scientific) software} using a
        particular \emph{compiler toolchain}, as specified by one or multiple
        \emph{easyconfig files}.
    \end{block}
\end{minipage}
\end{frame}

%===============================================================================

\begin{frame}{EasyBuild: feature highlights}
\begin{itemize}
    \item
        Fully autonomously building and installing (scientific) software
        \begin{itemize}
            \item
                Automatic dependency resolution
            \item
                Automatic generation of modulefiles
        \end{itemize}
    \item
        Thorough logging of executed procedure
    \item
        Highly configurable, via config files/environment/command line
    \item
        Dynamically extendable with additional easyblocks, toolchains, etc.
    \item
        \emph{Support for module hierarchies}, via custom module naming scheme
    \item
        Supports over 500 different software packages
        \begin{itemize}
            \item
                Including CP2K, NAMD, NWChem, OpenFOAM, PETSc, QuantumESPRESSO,
                WRF, \ldots
        \end{itemize}
\end{itemize}
\quad\quad
\begin{minipage}{0.9\textwidth}
    \begin{exampleblock}{Example: building/installing WRF \& dependencies \emph{with a single command} (!)}
        \ttfamily
        \$ eb WRF-3.5-goolf-1.6.10.eb --robot\\
        \$ module load WRF/3.5-goolf-1.6.10
    \end{exampleblock}
\end{minipage}
\end{frame}

%===============================================================================

\begin{frame}{Lmod: a modern modules tool}
    Lmod empowers users to \textbf{control their working environment} while improving the user
    experience of interacting with module files, without hindering experts.
    \begin{itemize}
        \item \small{(almost)} drop-in alternative for Tcl-based module tools
        \item written in Lua, available since Oct'08
        \item developed by Robert McLay (TACC, UT Austin)
        \item frequent releases, driven by community demands and feedback
    \end{itemize}
    \begin{minipage}{0.9\textwidth}
    \begin{exampleblock}{Example: swapping modules using \texttt{ml} shorthand in a module hierarchy}
        \ttfamily
        \$ ml\\
        Currently loaded modules:\\
        ~1) GCC/4.8.2 \quad 2) MPICH/3.1.1 \quad 3) FFTW/3.3.2\\
        \$ ml -GCC Clang\\
        The following have been reloaded:\\
        ~1) FFTW/3.3.2 \quad 2) MPICH/3.1.1\\
        \$ ml\\
        Currently loaded modules:\\
        ~1) Clang/3.4 \quad 2) MPICH/3.1.1 \quad 3) FFTW/3.3.2
    \end{exampleblock}
    \end{minipage}
\end{frame}

%===============================================================================

\begin{frame}{Lmod: feature highlights}
    \begin{itemize}
        \item \emph{module hierarchy-aware} design and functionality
        \begin{itemize}
            \item searching across entire module tree with \texttt{spider} subcommand
            \item automatic reloading of dependent modules on \texttt{module swap}
        \end{itemize}
        \item caching of module files, for responsive subcommands (e.g., \texttt{avail})
        \item customizable behavior via provided hooks
        \item \texttt{ml} shorthand command, load/unload shortcuts
        \item various advanced features, including:
        \begin{itemize}
            \item defining module families
            \item assigning properties to modules
            \item stack-based definition of environment variables (\texttt{pushenv})
            \item user-definable collections of modules
        \end{itemize}
    \end{itemize}
\end{frame}

%===============================================================================

\begin{frame}{EasyBuild and Lmod communities}
\begin{itemize}
    \item bar
\end{itemize}
\end{frame}

%===============================================================================

\begin{frame}{Synergy between EasyBuild and Lmod}
\begin{itemize}
    \item bar
\end{itemize}
\end{frame}

%===============================================================================

\begin{frame}{Future work}
\begin{itemize}
    \item bar
\end{itemize}
\end{frame}

%===============================================================================

\begin{frame}{Related work}
\begin{itemize}
    \item bar
\end{itemize}
\end{frame}

%===============================================================================

\begin{frame}{Conclusion}
\begin{itemize}
    \item bar
\end{itemize}
\end{frame}

\end{document}
